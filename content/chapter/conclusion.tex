\chapter{Conclusion}

\subsubsection{Summary}

\subsubsection{Outlook}

It remains unclear if the \gls{aod} is designed to have a homogeneous
intensity transmission when supplied with sufficient power or intensity
irregularities arise in a complex interaction between acousto-optic and
electronic defects. Our experimental results support the first case but
evidence can be questioned and one shall not fall for hastly judgement.
One possibility to answer this question once and for all would be to use a
precision watt meter together with the direct-coupler we already used in XX
and to manually control a power level over a set of frequencies. The
resulting intensities should then indicate whether a homogeneous intensity
field is feasible (if necessary with calibration procedure) or not.

Regardless we have found that it is a challenging undertaking to maintain a
constant high-power \gls{rf} signal over a frequency range. The available
\gls{dds} and power amplifier do clearly not meet these demands. The
\gls{dds} also fail to provide sufficient frequency resolution on time scales
necessary for the practical use in ultracold atom experiments. However it is
likely that a custom \gls{fpga} with power amplification may be suited to
face such challenge.

Before that though we need to dig deeper into the theory of acousto-optics
and check if the acousto-optics can handle the targeted time scales or if
the laser beam will be subject to acoustic lensing effects. Literature that
covers these and more can be found with~\cite{Goutzoulis1994}
and~\cite{Royer1999}.

Finally there are also interesting industrial application to consider if the
outstanding issues can be resolved. For example the field of ophthalmologics
covers a wide range of opportunities for precision laser control like
corrective eye surgeries which according to~\cite{US20180110655},
~\cite{US20180064579} and~\cite{US7131968} make use of mechanical deflectors.
Certainly there are also industrial applications in solid-state industries,
yet we did not check up closer.
