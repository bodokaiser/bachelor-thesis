\chapter{Conclusion}


\subsubsection{Summary}

\subsubsection{Outlook}

It remains open if the \gls{aod} is designed to have a homogeneous intensity
transmission when supplied with sufficient power or intensity irregularities
arise in a complex interaction between acousto-optic and electronic defects.
One possibility to answer this question once and for all would be to use a
precision watt meter together with the direct-coupler we already used in XX
and to manually control a power level over a set of frequencies. The
resulting intensities should indicate whether a homogeneous intensity field
is feasible (if necessary with calibration procedure) or not. Our
experimental results strongly support the first case, however you shall never
fall for hastly judgements.

Anyhow we have found that it is a challenging undertaking to maintain a
constant high-power \gls{rf} signal over a frequency range that the given
\gls{dds} and power amplifier are not suited to face.


the resulting intensity. As long as the so measured intensity is
deterministic with respect to the applied radio frequency, one can try to
recalibrate the power level to yield a homogeneous intensity measurement as
we have done in the previous chapter.
