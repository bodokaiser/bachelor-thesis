\chapter{Radio frequency signal}


The present section considers the \gls{rf} signal we apply to the
acousto-optic deflectors at different stages of the electronic processing.

\section{Signal synthesis}

As already described in the experimental setup section we use \gls{dds} for
signal synthesis. We configure the \gls{dds} to do a linear frequency sweep
from $f_0=\SI{80}{\mega\hertz}$ to $f_1\SI{120}{\mega\hertz}$ as this will be
the later operation range to deflect the laser beam through the \gls{aod}.

\begin{figure}[ht]
  \centering
  \includegraphics[width=\textwidth]{\figuredir{signal/synthesis/spectrogram.pdf}}
  \captionsetup{width=.8\textwidth}
  \caption{Delayed spectrogram windows of the \gls{dds} output signal in linear
    ramp operation. We see how the frequency starts at $f_0$ and increases
  in discrete steps to the final frequency $f_1$.}
  \label{fig:signal_synthesis_spectrogram}
\end{figure}

At a configured timescale of \SI{10}{\micro\second} the used oscilloscopes
voltage resolution is sufficient for Fourier analysis of the data.
Unfortunately the oscilloscope is not able to capture the complete sweep
period of $T=\SI{26.85}{\milli\second}$. To overcome this obstacle we
inserted a frequency generator between the external trigger source and the
oscilloscope. The frequency generator is configured to emit a square wave
pattern on the rising edge of the external trigger of width $d$. The
oscilloscope was configured to be triggered on the falling edge of the
external trigger signal supplied by the frequency generator. By adjusting
the square wave width $d$ we effectively added a delay to the trigger signal.
In order to capture now the complete signal over duration $T$ we incremented
the delay $d$ in steps of $T/300$. \Cref{fig:signal_synthesis_spectrogram}
shows the spectrogram at four such delays. We observe that because of the
discrete nature of the operation of the \gls{dds} each window only compromises
one distinct frequency that is incremented over the sweep period $T$ in
discrete steps.

\begin{figure}[ht]
  \centering
  \includegraphics[width=\textwidth]{\figuredir{signal/synthesis/frequency-max-amplitude.pdf}}
  \captionsetup{width=.8\textwidth}
  \caption{Maximum amplitude of the two \gls{dds} at different frequencies
  during linear sweep operation. In the two lower curves the \gls{dds} were
configured with enabled inverse sinc filter that is supposed to compensate
frequency dependence in the amplitude.}
  \label{fig:signal_synthesis_frequency_max_amplitude}
\end{figure}

To find the frequency dependence of the amplitude we carried out \gls{fft} to
find the dominant frequency of the signal. Then we collected the maximum
amplitude of each signal window together with the dominant frequency and
visualized the result in \Cref{fig:signal_synthesis_frequency_max_amplitude}.
\Cref{fig:signal_synthesis_frequency_max_amplitude} shows us the maximum
amplitude frequency dependency for different \gls{dds} configurations. We have
used two different \gls{dds} chips as each will later supply a single
\gls{aod}. From \Cref{fig:signal_synthesis_frequency_max_amplitude} we
conclude that there is no difference between distinct \gls{dds}, however we
observe a strong frequency dependence of the amplitude. This behaviour seems
inherent for digital signal synthesis as the \gls{ad9910} comprises an
inverse sinc filter that is supposed to reduce these dependency. From
\Cref{fig:signal_synthesis_frequency_max_amplitude} we see that the effect
of the inverse sinc filter unfortunately only lowers the output amplitude but
does not smooth the frequency dependency.

We summarize that already at the signal source stage a significant frequency
dependency of the output power is introduced that will later have impact on
the effective beam deflection power in the \gls{aod}. Further in our
configuration the inverse sinc filter only lowered the output power, thus
we will keep it disabled for subsequent measurements.

\section{Signal amplification}

We now supply the \gls{dds} signal to the amplifier and feed its output
through attentuators to the oscilloscope. For the measurement procedure
we retain the delay window method described in the former section to capture
the complete trace.

\begin{figure}[ht]
  \centering
  \includegraphics[width=\textwidth]{\figuredir{signal/amplification/frequency-max-amplitude.pdf}}
  \captionsetup{width=.8\textwidth}
  \caption{Maximum amplitude at dominant frequency per delayed window for two
  different amplifiers. We can see the discreteness of the frequency domain
from the digital signal synthesis as well as three resonances. Further the
amplifier differ by a constant offset.}
  \label{fig:signal_amplification_frequency_max_amplitude}
\end{figure}

\Cref{fig:signal_amplification_frequency_max_amplitude} presents us the
damped output signal after amplification. Here we can clearly see the finite
frequencies issued by the \gls{dds} as horizontal lines. Further we see that
both amplifiers differ by a constant offset.

\begin{figure}[ht]
  \centering
  \includegraphics[width=\textwidth]{\figuredir{signal/amplification/comparison.pdf}}
  \captionsetup{width=.8\textwidth}
  \caption{Relative amplitude at dominant frequency per delayed window for
  two different amplifiers and their digital signal source. In comparison to
the signal source the amplifiers introduce a further resonance at the
center frequency.}
  \label{fig:signal_amplification_comparison}
\end{figure}

For \Cref{fig:signal_amplification_comparison} we compared the normalized
amplified signals with the normalized signal source. We observe similar
signal shape for both amplifiers. Compared to the signal source frequency
response has changed in that there is an additional frequency response in
between of the two responses of the signal source.

\section{Signal reflection}

In the previous section we found.

\begin{figure}[ht]
  \centering
  \includegraphics[width=\textwidth]{\figuredir{signal/reflection/direct.pdf}}
  \captionsetup{width=.8\textwidth}
  \caption{We see the signal reflection of the two different \gls{aod} when
  connected directly to the network analyzer. We can see that both crystal
differ in their respective spectrum.}
  \label{fig:signal_reflection_direct}
\end{figure}

\begin{figure}[ht]
  \centering
  \includegraphics[width=\textwidth]{\figuredir{signal/reflection/coupler.pdf}}
  \captionsetup{width=.8\textwidth}
  \caption{Input power reflection when supplying the direct-coupler with
    \SI{-10}{\deci\bell\meter} input signal and reflection at the output of
    the direct-coupler while other ports are closed with \SI{50}{\ohm}}.
  \label{fig:signal_reflection_coupler}
\end{figure}

\begin{figure}[ht]
  \centering
  \includegraphics[width=\textwidth]{\figuredir{signal/reflection/coupled.pdf}}
  \captionsetup{width=.8\textwidth}
  \caption{Reflection at the direct-coupler output after amplification of the
  network analyzer input signal for different effective powers. We see that
the applied power does not effect the spectrum.}
  \label{fig:signal_reflection_coupled}
\end{figure}

\begin{figure}[ht]
  \centering
  \includegraphics[width=\textwidth]{\figuredir{signal/reflection/comparison.pdf}}
  \captionsetup{width=.8\textwidth}
  \caption{Comparison of reflection from amplified input signal and direct
  signal provided from the network analyzer. The different reflection spectrum
can be associated to the amplifier.}
  \label{fig:signal_reflection_comparison}
\end{figure}
