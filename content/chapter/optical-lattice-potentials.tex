\chapter{Optical lattice potentials}

In the introduction we presented the concept of our envisaged potential
creation. In the following chapter we want to recapitulate the theoretical
foundations of optical lattice potentials and how it alters under an
additional optical potential well.

\section{Atom-light interaction}

At first we need to ask ourselves how the laser field propagates as
perturbation potential to the Hamiltonian of the atomic system and how we can
express said potential in terms of the intensity distribution we can control
throughout the experiment.

\subsection{Dipole potential}

We will use \gls{si} units if not noted else. The Hamiltonian of an electron
in an external electromagnetic field reads
\begin{equation}
  \hat{H}_\text{em}
  =\frac{1}{2m}\left(\hat{\vb{p}}+e\vb{A}\right)^2-e\Phi+\hat{V}_0
  \label{eq:hamiltonian_em}
\end{equation}
with vector potential $\vb{A}$ and scalar potential $\Phi$ of the external
field and the Coulomb potential of the nucleus $\hat{V}_0$.
\Cref{eq:hamiltonian_em} is exact for the hydrogen atom that only hosts one
electron. For alkali atoms we know that inner electron shells are closed and
the single outer electron is approximately described by
\cref{eq:hamiltonian_em}. The electromagnetic field relates to the
vector and scalar potential through
\begin{equation}
  \vb{E}=-\grad{\Phi}-\pdv{\vb{A}}{t} &&
  \vb{B}=\curl{\vb{A}}
  \label{eq:field_em_potentials}
\end{equation}
where $\vb{E}$ is the electric and $\vb{B}$ the magnetic field component.

\subsubsection{Gauge freedom}

Gauge freedom of the electromagnetic vector and scalar potential permit
transformations of kind
\begin{align}
  \vb{A}\to\vb{A}+\grad{\chi}
  &&
  \Phi\to\Phi-\pdv{\chi}{t}
  \label{eq:gauge_transform_em}
\end{align}
with gauge function $\chi(\vb{x},t)$.

We can see that insertion of \cref{eq:gauge_transform_em} into
\cref{eq:field_em_potentials} and applying vector calculus identities yields
expressions for the electromagnetic field components independent of the gauge
function $\chi$. We usually only consider the electromagnetic field to
represent physical reality, thus Gauge freedom describes physical redunancy
in mathematically distinct solutions.\footnote{The Aharonov-Bohm effect
actually provides evidence that also the electromagnetic potentials represent
physical reality.}

\subsubsection{Dipole approximation}

In the following we choose the gauge function $\chi=-\vb{A}\cdot\vb{x}$ and
assume the dipole approximation $\vb{A}(\vb{x},t)\approx\vb{A}(t)$. The
dipole approximation is reasonable as wave length of visible light are much
larger then atomic length scales \cite{Gerry2004}. In the dipole
approximation $\chi$ satisfies the Coulomb gauge condition
$\divergence{\vb{A}}=0$ allowing us to set $\Phi=0$ as no external sources
are present \cite{Jackson2005}. Finally we can rewrite
\cref{eq:hamiltonian_em} as
\begin{equation}
  \hat{H}_\text{dip}
  =\frac{\hat{\vb{p}}^2}{2m}+\hat{V}_0+\hat{V}_\text{dip}
  =\hat{H}_0+\hat{V}_\text{dip}
  \label{eq:hamiltonian_dip}
\end{equation}
with the dipole potential
\begin{equation}
  \hat{V}_\text{dip}
  =-\hat{\vb{d}}\cdot\vb{E}
  \label{eq:potential_dip},
\end{equation}
dipole operator $\hat{\vb{d}}=-e\vb{x}$ and spatially homogeneous light
field $\vb{E}(t)$.

\subsection{Effective potential}

We are now going to solve \cref{eq:hamiltonian_dip} for an arbitrary light
field of the form
\begin{equation}
  \vb{E}(t,\vb{x})
  =\vb{E}_0(\vb{x})\cos(\omega t)
  \label{eq:light_field}
\end{equation}
where $\vb{E}_0(\vb{x})$ should be compatible with Maxwell's equations and
be approximately constant on atomic length scales to not infringe with the
dipole approximation. Further we need the laser frequency $\omega$ to be
far-off-resonant to the atomic transition frequencies.

\subsubsection{General case}

At $t<0$ the system is in the energy eigenstate $\ket{n}$ of the
unperturbated Hamiltonian $\hat{H}_0$
\begin{equation}
  \hat{H}_0\ket{n}
  =E_n\ket{n}
  =\hbar\omega_n\ket{n}
  \label{eq:eigenvalue_energy_unperturbated}.
\end{equation}
At $t>0$ the external light field appears instantaneous. The new state
$\ket{\psi}$ can be expanded in the complete base of the previous energy
eigenstates
\begin{equation}
  \ket{\psi}
  =\sum_nc_n(t)e^{-i\omega_nt}\ket{n}
  \label{eq:state_expansion_unperturbated}.
\end{equation}
Inserting \cref{eq:state_expansion_unperturbated} into the time-dependent
Schrödinger equation with dipole Hamiltonian \cref{eq:hamiltonian_dip} and
applying $\bra{m}e^{i\omega_mt}$ to the right hand side leads us to a set
of differential equations
\begin{equation}
  \dot{c}_m
  =-\frac{i}{\hbar}\sum_nc_n(t)e^{-i\omega_{nm}t}\bra{m}\hat{V}_\text{dip}\ket{n}
  \label{eq:differential_equation_population_dynamics}
\end{equation}
with $\omega_{nm}=\omega_n-\omega_m$. The dynamics of the perturbated system
are described by the time-dependent probablity amplitudes $c_n(t)$.
By using \cref{eq:light_field} we can rewrite the dipole transition matrix
elements
\begin{equation}
  \bra{m}\hat{V}_\text{dip}\ket{n}
  =\Omega_{nm}(\vb{x})\cos(\omega t)\hbar
  \label{eq:elements_dipole_transition_matrix}
\end{equation}
where we introduced the Rabi frequency
\begin{equation}
  \Omega_{nm}(\vb{x})
  =\bra{n}\hat{\vb{d}}\cdot\vb{E}_0(\vb{x})\ket{m}/\hbar
  \label{eq:rabi_frequency}.
\end{equation}
Explicit values for general dipole transition elements for one and two
electron systems can be found in \cite{Bethe1957}. Transition elements
vanishes for states with same parity, in particular $\Omega_{nn}=0$
\cite{Bartelmann2018}.

\subsubsection{Two state system}

From now on we assume a two state system that initially is only populated in
the ground state $c_g(0)=1,c_e(0)=0$. Under these circumstances the dynamics
described in \cref{eq:differential_equation_population_dynamics} simplify to
\begin{align}
  i\dot{c}_g=\Omega_0c_e(t)\cos(\omega t)e^{+i\omega_0 t} &&
  i\dot{c}_e=\Omega_0c_g(t)\cos(\omega t)e^{-i\omega_0 t}
  \label{eq:differential_equation_population_dynamics_two_state_system}
\end{align}
with $\Omega_0=\Omega_{ge}(\vb{x})$. Expansion of $\cos(\omega t)$ in terms
of the exponential function and dropping $e^{\pm i(\omega+\omega_0)t}$ yields
\begin{align}
  i\dot{c}_g\approx\frac{\Omega_0}{2}c_e(t)e^{+i\Delta\omega t} &&
  i\dot{c}_e\approx\frac{\Omega_0}{2}c_g(t)e^{-i\Delta\omega t}
  \label{eq:differential_equation_population_dynamics_two_state_system_rwa}.
\end{align}
The so-called rotating wave approximation is motivated by the fact that
oscillations of frequency $\omega+\omega_0$ are fast compared to changes in
the population dynamics and therefore vanish on average.

We now define $a_g=c_g$ and $a_e=c_ee^{i\Delta\omega t}$ and refine
\cref{eq:differential_equation_population_dynamics_two_state_system_rwa} by
\begin{align}
  i\dot{a}_g=\frac{\Omega_0}{2}a_e(t) &&
  i\dot{a}_e=\frac{\Omega_0}{2}a_g(t)-a_e\Delta\omega
  \label{eq:differential_equation_population_dynamics_two_state_system_shift}.
\end{align}
In the former form we can diagonalize the Hamiltonian and find energy
eigenvalues to be
\begin{equation}
  E_{e,g}
  =\frac{\hbar}{2}\left(-\Delta\omega\mp\sqrt{\Omega_0^2+\Delta\omega^2}\right)
  \approx
  \mp\frac{\hbar\Omega_0^2}{4\Delta\omega}
  \label{eq:eigenvalues_energy_light_shift}
\end{equation}
where we applied Taylor expansion to the square root for
$\Delta\omega\gg\Omega_0$.

\subsubsection{AC light shift}

In result atoms in an external off-resonant light field experience an
effective periodic dipole potential
\begin{equation}
  \hat{V}_\text{eff}(\vb{x})
  =
  \mp\frac{\hbar\Omega_0^2(\vb{x})}{4\Delta\omega}
  =
  \mp\frac{d_0^2E_0^2(\vb{x})}{4\hbar\Delta\omega}
  \label{eq:potential_effective}
\end{equation}
with dipole element $d_0$. Same results can also be obtained by the use of
second order perturbation theory, however the presented approach is more
clear on assumptions and time dependence \cite{Grimm2008}.

\section{Electrical field of laser}

An one dimensional optical lattices can be generated through the interference
of two counter-propagating Gaussian beams, creating a standing wave
interference pattern. The electrical field component in such a case is given
by
\begin{equation}
  \vb{E}(t)
  =\vb{E}_0\cos(kz)\cos(\omega t)\exp(-\frac{2r^2}{w(z)^2})
  \label{eq:1d_lattice_efield_exact},
\end{equation}
wherein $r$ denotes the beam radius, $w(z)$ the spot size parameter and $k$
the angular wavenumber of the laser. The wavenumber relates to the laser
wavelength via $k=2\pi/\lambda$. The magnetic field component is chosen
in such a way that it satisfies the Maxwell equations but will not be
considered relevant any further as the electron only has a small magnetic
dipole. Analogue to \cite{Rom2009} we will now assume $w(z)\gg r$ and drop
the exponential term in \cref{eq:1d_lattice_efield_exact}, thus simplifying
\cref{eq:1d_lattice_efield_exact} to
\begin{equation}
  \vb{E}(t)
  \approx\vb{E}_0\cos(kz)\cos(\omega t)
  \label{eq:1d_lattice_efield_approx}.
\end{equation}
Note that even the electric field in \cref{eq:1d_lattice_efield_approx} has
a spatial dependency, the dipole approximation still holds in the atomic
reference frame.

