\chapter{Optical lattices}

In the introduction we presented the concept of our envisaged potential
creation. In the following chapter we want to recapitulate the theoretical
foundations of optical lattices and quantum states theoreof. Finally we want
to derive an expression for the quantum tunnelling frequency in order to
estimate the time magnitude in which our apparatus has to operate.

\section{Atom-light interaction}

At first we need to ask ourselves how the laser field propagates as
perturbation potential to the Hamiltonian of the atomic system and how we can
express said potential by physical quantities we control in the experiment.

\subsection{Dipole approximation}

The Hamiltonian of an electron in an external electromagnetic field reads
\begin{equation}
  \hat{H}_\text{em}
  =\frac{1}{2m}\left(\hat{\vb{p}}+e\vb{A}\right)^2-e\Phi+\hat{V}_0
  \label{eq:hamiltonian_em}
\end{equation}
with vector $\vb{A}$ and scalar potential $\Phi$ of the external field and
the Coulomb potential of the nucleus $\hat{V}_0$. \Cref{eq:hamiltonian_em} is
exact for the hydrogen atom and approximate for alkali atoms with one outer
electron.

Gauge freedom of the vector and scalar potential permits transformations of
type
\begin{align}
  \vb{A}\to\vb{A}+\grad{\chi}
  &&
  \Phi\to\Phi-\pdv{\chi}{t}
\end{align}
with gauge function $\chi(\vb{x},t)$. In the following we choose the gauge
function $\chi=-\vb{A}\cdot\vb{x}$ and assume the dipole approximation
$\vb{A}(\vb{x},t)\approx\vb{A}(t)$. The dipole approximation is reasonable as
wave length of visible light are much larger then atomic length scales
\cite{Gerry2004}. In the dipole approximation $\chi$ satisfies the Coulomb
gauge condition $\divergence{\vb{A}}=0$ allowing us to set $\Phi=0$ as no
external sources are present \cite{Jackson2005}. Finally we can rewrite
\cref{eq:hamiltonian_em} as
\begin{equation}
  \hat{H}_\text{dip}
  =\frac{\hat{\vb{p}}^2}{2m}+\hat{V}_0-\hat{\vb{d}}\cdot\vb{E}
  =\hat{H}_0+\hat{V}_\text{dip}
  \label{eq:hamiltonian_dip}
\end{equation}
with the dipole operator $\hat{\vb{d}}=-e\vb{x}$ and spatially homogeneous
light field $\vb{E}(t)$.

\subsection{Effective dipole potential}

We are now going to solve \cref{eq:hamiltonian_dip} for the case of an
electric field of the form \cref{eq:1d_lattice_efield_approx} with
for-off-resonant laser wavelength $\lambda$ and finally derive an expression
for the effective dipole potential.

At $t<0$ the system is in the energy eigenstate $\ket{n}$ of the unperturbated
Hamiltonian
\begin{equation}
  \hat{H}_0\ket{n}
  =E_n\ket{n}
  =\hbar\omega_n\ket{n}
  \label{eq:eigenvalue_energy_unperturbated}.
\end{equation}
At $t>0$ the external light field appears instantaneous. The new state
$\ket{\psi}$ can be expanded in the complete base of the previous energy
eigenstates
\begin{equation}
  \ket{\psi}
  =\sum_nc_n(t)e^{-i\omega_nt}\ket{n}
  \label{eq:state_expansion_unperturbated}.
\end{equation}
Inserting \cref{eq:state_expansion_unperturbated} into the time-dependent
Schrödinger equation with dipole Hamiltonian \cref{eq:hamiltonian_dip} and
applying $\bra{m}e^{i\omega_mt}$ to the right hand side leads us to a set
of differential equations
\begin{equation}
  \dot{c}_m
  =-\frac{i}{\hbar}\sum_nc_n(t)e^{-i\omega_{nm}t}\bra{m}\hat{V}_\text{dip}\ket{n}
  \label{eq:differential_equation_population_dynamics}
\end{equation}
with $\omega_{nm}=\omega_n-\omega_m$, that describe the dynamics of the
probablity amplitudes $c_n(t)$. By using the electric field we derived in
\cref{eq:1d_lattice_efield_approx} we can rewrite the dipole transition
matrix elements
\begin{equation}
  \bra{m}\hat{V}_\text{dip}\ket{n}
  =\hbar\Omega_{nm}\cos(kz)\cos(\omega t)
  \label{eq:elements_dipole_transition_matrix}
\end{equation}
where we introduced the Rabi frequency
$\Omega_{nm}=\bra{n}\hat{\vb{d}}\cdot\vb{E}_0\ket{m}/\hbar$. Explicit values
for general dipole transition elements for one and two electron systems can
be found in \cite{Bethe1957}. Als the transition elements vanishes for states
with same parity, in particular do we have $\Omega_{nn}=0$
\cite{Bartelmann2018}.

\subsubsection{Two state system}

From now on we assume a two state system that initially is only populated in
the ground state $c_g(0)=1,c_e(0)=0$ and the dynamics in
\cref{eq:differential_equation_population_dynamics} are simply described by
\begin{align}
  i\dot{c}_g=\Omega_0c_e(t)\cos(kz)\cos(\omega t)e^{+i\omega_0 t} &&
  i\dot{c}_e=\Omega_0c_g(t)\cos(kz)\cos(\omega t)e^{-i\omega_0 t}
  \label{eq:differential_equation_population_dynamics_two_state_system}.
\end{align}
Expansion of $\cos(\omega t)$ in terms of the exponential function and
dropping $e^{\pm i(\omega+\omega_0)t}$ yields
\begin{align}
  i\dot{c}_g\approx\frac{\Omega_0}{2}c_e(t)\cos(kz)e^{+i\Delta\omega t} &&
  i\dot{c}_e\approx\frac{\Omega_0}{2}c_g(t)\cos(kz)e^{-i\Delta\omega t}
  \label{eq:differential_equation_population_dynamics_two_state_system_rwa}.
\end{align}
The so-called rotating wave approximation is motivated by the fact that
oscillations of frequency $\omega+\omega_0$ are fast compared to changes in
the population dynamics and therefore vanish on average.

We now define $a_g=c_g$ and $a_e=c_ee^{i\Delta\omega t}$ and refine
\cref{eq:differential_equation_population_dynamics_two_state_system_rwa} by
\begin{align}
  i\dot{a}_g=\frac{\Omega_0}{2}a_e(t)\cos(kz) &&
  i\dot{a}_e=\frac{\Omega_0}{2}a_g(t)\cos(kz)-a_e\Delta\omega
  \label{eq:differential_equation_population_dynamics_two_state_system_shift}.
\end{align}
In the former form we can diagonalize the Hamiltonian and find energy
eigenvalues to be
\begin{equation}
  E_{e,g}
  =\frac{\hbar}{2}\left(-\Delta\omega\mp\sqrt{\Omega_0^2+\Delta\omega^2}\right)
  \approx
  =\mp\frac{\hbar\Omega_0^2}{4\Delta\omega}
  \label{eq:eigenvalues_energy_light_shift}
\end{equation}
where we applied Taylor expansion to the square root for
$\Delta\omega\gg\Omega_0$.

In summary atoms in an external off-resonant light field with experience an
effective periodic dipole potential
\begin{equation}
  \hat{V}_\text{eff}
  =V_0\cos^2(kz)
  \qc
  V_0
  =\frac{\hbar\Omega}{4\Delta\omega}\cos^2(kz)
  =\frac{\hbar I_0\cos^2(kz)}{2c_0\epsilon_0\Delta\omega}
  \label{eq:potential_effective}
\end{equation}
with vacuum speed of light $c_0$, dielectric constant $\epsilon_0$ and
maximum intensity $I_0$. Same results can also be obtained by the use of
second order perturbation theory, however the presented approach is more
clear on assumptions and time dependence \cite{Grimm2008}.

\section{Electrical field of laser}

An one dimensional optical lattices can be generated through the interference
of two counter-propagating Gaussian beams, creating a standing wave
interference pattern. The electrical field component in such a case is given
by
\begin{equation}
  \vb{E}(t)
  =\vb{E}_0\cos(kz)\cos(\omega t)\exp(-\frac{2r^2}{w(z)^2})
  \label{eq:1d_lattice_efield_exact},
\end{equation}
wherein $r$ denotes the beam radius, $w(z)$ the spot size parameter and $k$
the angular wavenumber of the laser. The wavenumber relates to the laser
wavelength via $k=2\pi/\lambda$. The magnetic field component is chosen
in such a way that it satisfies the Maxwell equations but will not be
considered relevant any further as the electron only has a small magnetic
dipole. Analogue to \cite{Rom2009} we will now assume $w(z)\gg r$ and drop
the exponential term in \cref{eq:1d_lattice_efield_exact}, thus simplifying
\cref{eq:1d_lattice_efield_exact} to
\begin{equation}
  \vb{E}(t)
  \approx\vb{E}_0\cos(kz)\cos(\omega t)
  \label{eq:1d_lattice_efield_approx}.
\end{equation}
Note that even the electric field in \cref{eq:1d_lattice_efield_approx} has
a spatial dependency, the dipole approximation still holds in the atomic
reference frame.

