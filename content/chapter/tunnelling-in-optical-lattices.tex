\chapter{Tunnelling in optical lattices}

In the introduction we presented the concepts behind our enivisaged
generation of local time-averaged high-precision optical potentials. In this
chapter we want to recapitulate the theoretical foundations of optical
lattices and estimate theoretical boundaries for the time-average.

\section{Atom-field interactions}

The Hamiltonian of an electron in an external electromagnetic field reads
\begin{equation}
  \hat{H}_\text{em}
  =\frac{1}{2m}\left(\hat{\vb{p}}+e\vb{A}\right)^2-e\Phi+\hat{V}
  \label{eq:hamiltonian_em}
\end{equation}
with vector $\vb{A}$ and scalar potential $\Phi$. The Coulomb potential of
the nucleus is represented by the operator $\hat{V}$. For an atom interaction
with an external light field we can use the dipole approximation
$\vb{A}(\vb{x},t)=\vb{A}(t)$ \cite{Gerry2004}[p.76].

Gauge freedom permits transformations of type
\begin{align}
  \vb{A}\to\vb{A}+\grad{\chi}
  &&
  \Phi\to\Phi-\pdv{\chi}{t}
\end{align}
with gauge function $\chi$. In the following we choose the gauge function
$\chi=-\vb{A}\cdot\vb{x}$. In the dipole approximation $\chi$ satisfies the
Coulomb gauge condition $\divergence{\vb{A}}=0$ allowing us to set $\Phi=0$
as no external sources are present \cite{Jackson2005}[p.242]. Finally we can
write \cref{eq:hamiltonian_em} as
\begin{equation}
  \hat{H}_\text{dip}
  =\frac{\hat{\vb{p}}^2}{2m}+\hat{V}-\hat{\vb{d}}\cdot\vb{E}
  \label{eq:hamiltonian_dip}
\end{equation}
with the dipole operator $\hat{\vb{d}}=-e\vb{x}$ and spatially homogeneous
light field $\vb{E}(t)$.

\subsection{Classical field}


% conservative
\subsubsection{Rabi model}
\subsubsection{Effective potential}

\subsection{Quantized field}

For our use case the classical treatment of light is sufficient. For a
a quantized light field $\vb{E}(t)$ we will find nonzero probabilities
for transitions between energy levels, known as spontaneous emissions,
for which an elaborate treatment can be found in \cite{Gerry2004}.

\section{Lattice band structure}

\subsection{Schrödinger equation}

The Schrödinger equation of a particle state $\ket{\Psi}$ subject to a time
independent one-dimensional potential reads
\begin{align}
  E\ket{\Psi}
  =\hat{H}\ket{\Psi}
  =\left(\frac{\hat{p}^2}{2m}+\hat{V}\right)\ket{\Psi}
  \label{eq:time_independent_schroedinger}.
\end{align}
For a periodic potential $V(x+a)=V(x)$ the Bloch-Theorem \cite{Ashcroft1976}
states that the wavefunction of a particle in a periodic potential can be
written as
\begin{equation}
  \braket{\hat{x}}{\Psi}
  =\Psi(x)
  =e^{iqx}\psi(x)
  \label{eq:bloch_theorem}
\end{equation}
wherein $-\frac{\pi}{a}<q<+\frac{\pi}{a}$ is the quasimomentum, a quantum
number characteristic of the translational symmetry of the periodic potential
\cite[p. 42]{Lewenstein2012} with $a$ being the lattice constant.

The momentum operator is a differential operator in position space
\begin{equation}
  \hat{p}
  =\expval{\hat{p}}{\hat{x}}
  =-i\hbar\dv{x}
  \label{eq:momentum_operator_in_position_space},
\end{equation}
thus we obtain $\hat{p}e^{iqx}=\hbar qe^{iqx}$ and
$\hat{p}^2\ket{\Psi}=e^{iqx}\left(\hat{p}+\hbar q\right)^2\ket{\psi}$ such
that we can rewrite \cref{eq:time_independent_schroedinger} for
the periodic case
\begin{equation}
  E\ket{\psi}
  =\left(\frac{(\hat{p}+\hbar q)^2}{2m}+\hat{V}\right)\ket{\psi}
  =:\hat{H}_{periodic}\ket{\psi}
  \label{eq:periodic_schroedinger}.
\end{equation}

Translational symmetry of the potential enforces discrete particle momenta
$p_n=pn=nk\hbar$ such that we can represent our particle state in a
countable momentum basis
\begin{equation}
  \ket{\psi}
  =\mathbb{1}\ket{\psi}
  =\left(\sum_n\ket{p_n}\bra{p_n}\right)\ket{\psi}
  =\sum_n\braket{p_n}{\psi}\ket{p_n}
  =\sum_nc_n\ket{p_n}
  \label{eq:countable_momentum_state}.
\end{equation}
Inserting this into the right hand side of \cref{eq:periodic_schroedinger}
yields us
\begin{equation}
  \hat{H}_\text{periodic}\ket{\psi}
  =\sum_nc_n\left(\frac{(\hat{p}+\hbar q)^2}{2m}+\hat{V}\right)\ket{p_n}
  =\sum_nc_n\left(\frac{(pn+\hbar q)^2}{2m}+\hat{V}\right)\ket{p_n}
  \label{eq:periodic_hamiltonian_in_momentum_base}
\end{equation}
where we used the eigenvalue equation $\hat{p}\ket{p_n}=pn\ket{p_n}$ to
satisfy the second equal sign. By applying an orthogonal $\bra{p_m}$ to the
left hand side of \cref{eq:periodic_hamiltonian_in_momentum_base} we find
the matrix elements of the periodic Hamiltonian
\begin{equation}
  \hat{H}_{\text{periodic},mn}
  =\mel**{p_m}{\hat{H}_\text{periodic}}{p_n}
  =(m+q/k)^2\frac{\hbar^2k^2}{2m}+\mel**{p_m}{\hat{V}}{p_n}
  \label{eq:periodic_hamiltonian_elements}
\end{equation}
where we used $p=\hbar k$ in the last step.

Usually the potential will be known in position space. Through use of the
completeness relation
\begin{equation}
  \mathbb{1}
  =\int\dd{x}\op{x}{x}
  \label{eq:continous_completeness_relation}
\end{equation}
we are able to transform the potential in momentum space
\begin{equation}
  V(p)
  =\mel**{p_m}{\hat{V}}{p_n}
  =\int\int\dd{x}\dd{y}\braket{p_m}{x}\mel**{x}{\hat{V}}{y}\braket{y}{p_n}
  =\int\frac{\dd{x}}{\sqrt{2\pi\hbar}}e^{ikx(m-n)}V(x)
  \label{eq:potential_fourier_transform}.
\end{equation}
after the identification of $\braket{p}{x}=e^{ipx/\hbar}/\sqrt{2\pi\hbar}$.

\subsection{Bloch theorem}

\subsection{Band structure}

In the former section we discussed the quantum mechancis of a generic
periodic potential. We will now consider the special case of a harmonic
potential of the shape
\begin{equation}
  V(x)
  =V_0\cos^2(kx)
  =\frac{1}{4}V_0\left(e^{+2ikx}+e^{-2ikx}+2\right)
  \label{eq:harmonic_potential}.
\end{equation}
Potentials of such form can be found for example in the standing wave of a
reflected gaussian beam
\begin{equation}
  V(r,x)
  =V_0e^{-2r^2/w(x)^2}\cos^2(kx)
  \approx V_0\cos^2(kx)
  \label{eq:optical_potential}
\end{equation}
wherein $r$ denotes the beam radius, $w(x)$ being spot size parameter and
$k=2\pi/\lambda$ with the laser wavelength $\lambda$. The approximation is
valid for $w(x)\gg r$ which holds in most cases according to
\cite[p.127]{Rom2009}.

Transformation of \cref{eq:harmonic_potential} to momentum space
using \cref{eq:potential_fourier_transform} yields
\begin{equation*}
  V(p)
  =
  \frac{1}{4}V_0
  \int\frac{\dd{x}}{\sqrt{2\pi\hbar}}
  e^{+ipx(m-n)/\hbar}\left(e^{+2ikx}+e^{-2ikx}+2\right)
  =
  \frac{1}{4}V_0\left(2\delta_{m,n}+\delta_{m,n+2}+\delta_{m,n-2}\right).
\end{equation*}
The Hamiltonian \cref{eq:periodic_hamiltonian_elements} of a
harmonic potential then reads
\begin{equation}
  \hat{H}_\text{harmonic}
  =\begin{cases}
    (n+q/k)^2\frac{\hbar^2k^2}{2m}+\frac{1}{2}V_0, & \text{if } n=m\\
    \frac{1}{4}V_0, & \text{if } \abs{n-m}=2\\
    0, & \text{otherwise}
  \end{cases}
  \label{eq:harmonic_hamiltonian}.
\end{equation}
Plotting the eigenvalues of \cref{eq:harmonic_hamiltonian} against
$q$ we find the characteristic energy band structure giving us the allowed
energy ranges.

In the case of optical potentials. We would further introduce the recoil
energy $E_r=(\hbar k)^2/(2m)$ which gives us a natural atom independent scale
to work with.

\subsection{Harmonic approximation}

Cooled atoms have a low kinetic energy and will be located at the ground
state which allows us to approximate
\begin{equation}
  V(x)
  =V_0\cos^2(kx)
  \approx V_0\left(1-(kx)^2\right)
  \label{eq:harmonic_approximation}
\end{equation}
inserting \cref{eq:harmonic_approximation} into
\cref{eq:time_independent_schroedinger} and rearraning yield us
\begin{equation}
  \hbar\omega(n+\frac{1}{2})\ket{\phi}
  =(E_n-V_0)\ket{\phi}
  =\left(\frac{\hat{p}^2}{2m}-V_0k^2x^2\right)\ket{\phi}
  =\hat{H}_\text{harmonic}\ket{\phi}
\end{equation}
with $\omega=\sqrt{V_0/m}k=\sqrt{2V_0E_r}/\hbar$.
