\chapter{Tunnelling in optical lattices}

In the introduction we presented the concepts behind our enivisaged
generation of local time-averaged high-precision optical potentials. In this
chapter we want to recapitulate the theoretical foundations of optical
lattices and estimate theoretical boundaries for the time-average.

\section{Atom-field interactions}

The Hamiltonian of an electron in an external electromagnetic field reads
\begin{equation}
  \hat{H}_\text{em}
  =\frac{1}{2m}\left(\hat{\vb{p}}+e\vb{A}\right)^2-e\Phi+\hat{V}_0
  \label{eq:hamiltonian_em}
\end{equation}
with vector $\vb{A}$ and scalar potential $\Phi$. The Coulomb potential of
the nucleus resides inside the potential operator $\hat{V}_0$. For an atom
interaction with an external light field we can apply the dipole approximation
$\vb{A}(\vb{x},t)\approx\vb{A}(t)$ \cite{Gerry2004}[p.76].

Gauge freedom permits transformations of type
\begin{align}
  \vb{A}\to\vb{A}+\grad{\chi}
  &&
  \Phi\to\Phi-\pdv{\chi}{t}
\end{align}
with gauge function $\chi$. In the following we choose the gauge function
$\chi=-\vb{A}\cdot\vb{x}$. In the dipole approximation $\chi$ satisfies the
Coulomb gauge condition $\divergence{\vb{A}}=0$ allowing us to set $\Phi=0$
as no external sources are present \cite{Jackson2005}[p.242]. Finally we can
write \cref{eq:hamiltonian_em} as
\begin{equation}
  \hat{H}_\text{dip}
  =\frac{\hat{\vb{p}}^2}{2m}+\hat{V}_0-\hat{\vb{d}}\cdot\vb{E}
  =\frac{\hat{\vb{p}}^2}{2m}+\hat{V}_0+\hat{V}_\text{dip}
  \label{eq:hamiltonian_dip}
\end{equation}
with the dipole operator $\hat{\vb{d}}=-e\vb{x}$ and spatially homogeneous
light field $\vb{E}(t)$.

\subsection{Classical field}

In the quantum treatment of light we would find nonzero probabilities for
transitions between energy levels without the presence of an external light
field, known as spontaneous emissions \cite{Gerry2004}. These dissipative
effects are among others relevant for various laser cooling techniques and
the study of population dynamics where we the laser light is near resonance.
In our use case the classical treatment of light is sufficient as we are
interested in the effective potential seen by the atoms from an
far-off-resonant laser.

\subsubsection{One dimensional lattice}

A one dimensional optical lattices can be generated by creating a standing
wave interference pattern by reflection of a single Gaussian laser beam. The
electrical field component in such a case is given by
\begin{equation}
  \vb{E}(t)
  =\vb{E}_0\cos(kz)\cos(\omega t)e^{-2r^2/w(z)^2}
  \label{eq:1d_lattice_efield_exact},
\end{equation}
wherein $r$ denotes the beam radius, $w(z)$ being spot size parameter and
$k=2\pi/\lambda$ with the laser wavelength $\lambda$. Analogue to
\cite[p.127]{Rom2009} we assume $w(z)\gg r$ and drop the exponential,
simplifying \cref{eq:1d_lattice_efield_exact} to
\begin{equation}
  \vb{E}(t)
  \approx\vb{E}_0\cos(kz)\cos(\omega t)
  \label{eq:1d_lattice_efield_approx}.
\end{equation}
Note that the dipole approximation still holds for
\cref{eq:1d_lattice_efield_approx} as for each atom the $z$ coordinate
will be approximately constant.

\subsubsection{Second order perturbation}

In the case of potential generation for ultracold atoms, we want to avoid any
population dynamics and keep the atoms in their respective ground state.
Therefore our optical potential is created with a far-off-resonance laser
frequency. Second order perturbation theory predicts an energy shift of
\begin{equation}
  \Delta E_n
  \approx
  \mel{\psi_n}{\hat{V}_\text{dip}}{\psi_n}
  +
  \sum_{m\neq n}\frac{\abs{\mel{\psi_m}{\hat{V}_\text{dip}}{\psi_n}}^2}{E_n-E_m}
  \label{eq:second_order_perturbation}.
\end{equation}
Symmetry of the atomic wave functions only allows nonzero dipole moments for
state transitions, therefore $\mel{\psi_n}{\hat{V}_\text{dip}}{\psi_n}=0$.
Further we assume only the first two energy levels to be relevant for
ultracold atoms, yielding us a two-level system and simplifying
\cref{eq:second_order_perturbation} to
\begin{equation}
  \Delta E
  \approx
  \pm\frac{\abs{\mel{\psi_0}{\hat{V}_\text{dip}}{\psi_1}}^2}{\abs{E_0-E_1}}
  =
  \pm\frac{c\epsilon_0 \abs{d_{01}}^2}{2\hbar\delta}I_0\cos^2(kz)
  \label{eq:light_shift}
\end{equation}
with detuning $\delta=\abs{\omega_{01}-\omega}$, maximum laser intensity
$I_0$, dielectric constant $\epsilon_0$, vacuum speed of light $c$ and dipole
moment $d_{01}$ for the state transition from ground to first excited state
with linear polarized light. The sign in \cref{eq:light_shift} has to be
chosen appropriate to the direction of the laser detuning. For red detuned
light the negative sign has to be choosen and atoms will be confined to the
nodes of the standing light wave. Explicit values for general dipole
transition elements for one and two electron systems can be found in
\cite{Bethe1957}.

For the upcoming sections we will take the effective potential in the reference
frame of atoms in an one dimensional optical lattice to be
\begin{equation}
  V_\text{lat}(z)=V_0\cos^2(kz)
  \label{eq:potential_lattice}
\end{equation}
with $V_0<0$ for red detuned light.

\section{Lattice band structure}

With the previous foundations in place we can proceed with the solid state
analogues for optical lattices. From the  the simple analytic form of the
optical potential, \cref{eq:optical_lattice_potential}, we are able to
derive an expression for the quantum tunneling probability which gives us
an upper bound for the duration time of our potential generation.

\subsection{Bloch theorem}

Bloch's theorem states that for a periodic potential
\begin{equation}
  V_\text{per}(\vb{x}+\vb{a})=V_\text{per}(\vb{x})
  \label{eq:potential_periodic}
\end{equation}
there exists a complete set of wavefunctions that are energy eigenstates
of the Hamiltonian
\begin{equation}
  \hat{H}_\text{per}
  =\frac{\hat{\vb{p}}^2}{2m}+\hat{V}_\text{per}
  \label{eq:hamiltonian_periodic},
\end{equation}
and each of these Bloch waves can be written in the form
\begin{equation}
  \bra{\vb{x}}\ket{\Psi_{n\vb{q}}}
  =\Psi_{n\vb{q}}(\vb{x})
  =e^{i\vb{q}\cdot\vb{x}}\psi_{n\vb{q}}(\vb{x})
  \label{eq:wave_bloch}
\end{equation}
with the wave vector $\vb{q}$ confined to the first Brillouin zone,
$\psi_{n\vb{q}}(\vb{x}+\vb{a})=\psi_{n\vb{q}}(\vb{x})$ and band index number
$n$. For additional information see \cite{Roessler2004}[p.126] or
\cite{Bartelmann2018}[p.897].

\subsection{Band structure}

We are now going to find the energy eigenvalues of the Hamiltonian
\begin{equation}
  \hat{H}_\text{lat}
  =\frac{\hat{\vb{p}}}{2m}+\hat{V}_\text{lat}
  \label{eq:hamiltonian_lattice}
\end{equation}
with $V_\text{lat}(z)=V_0\cos^2(kz)$ from \cref{eq:potential_periodic}. As
ansatz we use the Bloch waves \cref{eq:wave_bloch} and note that by the
product rule
$\hat{p}^2\ket{\Psi_{nq}}=e^{iqx}(\hat{p}+\hbar q)\ket{\psi_{nq}}$,
thus yielding
\begin{equation}
  E_{nq}\ket{\Psi_{nq}}
  =\hat{H}_{lat}\ket{\Psi_{nq}}
  =\left(\frac{(\hat{p}+\hbar q)^2}{2m}+\hat{V}_\text{lat}\right)\ket{\psi}
  \label{eq:eigenvalue_energy_lattice}.
\end{equation}

Through the completeness relation we can expand our state $\ket{\psi_{nq}}$
in the discrete crystal base $\ket{q_n}$
\begin{equation}
  \ket{\psi_{nq}}
  =\left(\sum_m\ket{k_m}\bra{k_m}\right)\ket{\psi_{nq}}
  =\sum_m\braket{k_m}{\psi_{nq}}\ket{k_m}
  =\sum_mc_m\ket{p_m}
  \label{eq:state_expansion_crystal_base}.
\end{equation}
Inserting this into the right hand side of
\cref{eq:eigenvalue_energy_lattice} yields us
\begin{equation}
  E_\text{lat}\ket{\psi_{nq}}
  \hat{H}_\text{lat}\ket{\psi_{nq}}
  =\sum_mc_n\left(\frac{(km+\hbar q)^2}{2m}+\hat{V}_\text{lat}\right)\ket{k_m}
  \label{eq:hamiltonian_lattice_crystal_base}
\end{equation}
where we used the eigenvalue equation $\hat{p}\ket{k_m}=km\ket{k_m}$ to
satisfy the second equal sign. By applying an orthogonal $\bra{k_l}$ to the
left hand side of \cref{eq:hamiltonian_lattice_crystal_base} we find
the matrix elements of the lattice Hamiltonian
\begin{equation}
  \hat{H}_{\text{lat},ml}
  =\mel**{p_l}{\hat{H}_\text{lat}}{p_m}
  =(m+q/k)^2\frac{\hbar^2k^2}{2m}+\mel**{p_m}{\hat{V}}{p_l}
  \label{eq:hamiltonian_lattice_element}
\end{equation}
where we used $p=\hbar k$ in the last step.


Usually the potential will be known in position space. Through use of the
completeness relation
\begin{equation}
  \mathbb{1}
  =\int\dd{x}\op{x}{x}
  \label{eq:continous_completeness_relation}
\end{equation}
we are able to transform the potential in momentum space
\begin{equation}
  V(p)
  =\mel**{p_m}{\hat{V}}{p_n}
  =\int\int\dd{x}\dd{y}\braket{p_m}{x}\mel**{x}{\hat{V}}{y}\braket{y}{p_n}
  =\int\frac{\dd{x}}{\sqrt{2\pi\hbar}}e^{ikx(m-n)}V(x)
  \label{eq:potential_fourier_transform}.
\end{equation}
after the identification of $\braket{p}{x}=e^{ipx/\hbar}/\sqrt{2\pi\hbar}$.
\begin{equation}
  V(x)
  =V_0\cos^2(kx)
  =\frac{1}{4}V_0\left(e^{+2ikx}+e^{-2ikx}+2\right)
  \label{eq:potential_dipole_crystal_base}.
\end{equation}

Transformation of \cref{eq:potential_lattice} to momentum space
using \cref{eq:potential_fourier_transform} yields
\begin{equation*}
  V(p)
  =
  \frac{1}{4}V_0
  \int\frac{\dd{x}}{\sqrt{2\pi\hbar}}
  e^{+ipx(m-n)/\hbar}\left(e^{+2ikx}+e^{-2ikx}+2\right)
  =
  \frac{1}{4}V_0\left(2\delta_{m,n}+\delta_{m,n+2}+\delta_{m,n-2}\right).
\end{equation*}
The Hamiltonian \cref{eq:hamiltonian_periodic_element} of a
harmonic potential then reads
\begin{equation}
  \hat{H}_\text{harmonic}
  =\begin{cases}
    (n+q/k)^2\frac{\hbar^2k^2}{2m}+\frac{1}{2}V_0, & \text{if } n=m\\
    \frac{1}{4}V_0, & \text{if } \abs{n-m}=2\\
    0, & \text{otherwise}
  \end{cases}
  \label{eq:harmonic_hamiltonian}.
\end{equation}
Plotting the eigenvalues of \cref{eq:hamiltonian_harmonic} against
$q$ we find the characteristic energy band structure giving us the allowed
energy ranges.

In the case of optical potentials. We would further introduce the recoil
energy $E_r=(\hbar k)^2/(2m)$ which gives us a natural atom independent scale
to work with.

\subsection{Harmonic approximation}

Cooled atoms have a low kinetic energy and will be located at the ground
state which allows us to approximate
\begin{equation}
  V(x)
  =V_0\cos^2(kx)
  \approx V_0\left(1-(kx)^2\right)
  \label{eq:potential_lattice_approximation}
\end{equation}
inserting \cref{eq:potential_lattice_approximation} into
\cref{eq:} and rearraning yield us
\begin{equation}
  \hbar\omega(n+\frac{1}{2})\ket{\phi}
  =(E_n-V_0)\ket{\phi}
  =\left(\frac{\hat{p}^2}{2m}-V_0k^2x^2\right)\ket{\phi}
  =\hat{H}_\text{harmonic}\ket{\phi}
\end{equation}
with $\omega=\sqrt{V_0/m}k=\sqrt{2V_0E_r}/\hbar$.
