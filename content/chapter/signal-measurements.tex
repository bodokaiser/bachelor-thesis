\chapter{Signal measurements}

By the time the \gls{rf} signal has reached the piezoelectric of the
\gls{aod} it has been synthesized from a reference signal, amplified to match
the power requirements of the \gls{aod} and matched to the impedance of the
\gls{aod}. Each of these stages ammends the shape of the \gls{rf} signal and
as we will see introduce new frequency responses of the amplitude. The so
inevitable amplitude modulation affects the deflected intensity of the laser
beam and will lead to the intensity characteristics discussed in the next
chapter.

\section{Digital signal synthesis}

As has been described in the experimental setup section, \gls{dds} are used
for signal synthesis. In the present section we want to examine the
frequency behaviour of the \gls{dds}. In particular we are interested how the
amplitude responds to different frequencies and how the digital design of the
\gls{dds} affects the \gls{rf} signal shape.

\subsection{Setup}
\label{subsec:signal_synthesis_setup}

For the following experiments we configured the \gls{dds} to do a linear
frequency sweep from \SI{80}{\mega\hertz} to \SI{120}{\mega\hertz} at a sweep
duration of \SI{26.84}{\micro\second}. The frequency range has been choosen
because it overlaps with the operation range required to drive the \gls{aod}.
The sweep duration is a good compromise between being able to resolve the
complete sweep with limited oscilloscope resolution on the one hand and
limited frequency resolution of the synthesizer.

The time resolution of most oscilloscopes depends on the selected time scale,
however choosing a finer time scale entails shortening the time length, thus
we have a trade-off between being able to perform signal analysis that demands
fine resolution of the sinusoidal voltage and the number of measurements we
need to perform to capture a complete passthrough of the synthesized signal
cycle. We found that a selected time division of \SI{10}{\micro\second} is
sufficient for signal analysis, i.e. \gls{fft}, but at the same time limits
the number of measurements to the magnitude of hundreds.

To overcome the obstacle that we can only capture one time window of a
\gls{rf} signal passthrough we inserted a frequency generator between the
external trigger source and the oscilloscope. The frequency generator is
configured to emit a square wave pattern on the rising edge of the external
trigger of width $d$. The oscilloscope was configured to be triggered on the
falling edge of the external trigger signal supplied by the frequency
generator. By adjusting the square wave width $d$ we effectively added a
delay to the trigger signal. In order to capture now the complete signal over
duration $T$ we incremented the delay $d$ in steps of $T/300$.

\subsection{Results}
\label{subsec:signal_synthesis_results}

The former setup produced a single dataset we evaluated from two different
angles. First we explored how the digital design of the \gls{dds} propagates
into the signal. Second we explored the frequency dependence of the amplitude.

\subsubsection{Spectrogram}

A spectrogram visualizes how the frequency spectrum varies in time. One way
to obtain a spectrogram is to partition the data into overlapping time chunks
while performing \gls{fft} which allows us to combine time and frequency
domain specific characteristics. In our case we choose the relative spectral
power to be encoded through color.

\begin{figure}[h]
  \centering
  \includegraphics[width=\textwidth]{\figuredir{signal/synthesis/spectrogram.pdf}}
  \captionsetup{width=.8\textwidth}
  \caption{Spectrogram of delayed time windows of the \gls{dds} output signal
    configured to perform a frequency sweep from \SI{80}{\mega\hertz} to
    \SI{120}{\mega\hertz}. For an ideal linear sweep we would expect a linear
    timeline of the frequency, instead we observe a discrete set of
    frequencies which reflects the digital nature of the \gls{dds}.}
  \label{fig:signal_synthesis_spectrogram}
\end{figure}

\Cref{fig:signal_synthesis_spectrogram} depicts four spectrogram, each taken
at a different time window of the frequency sweep passthrough. The first
spectrogram captures the start of the frequency sweep. We can see how for the
first \SI{200}{\micro\second} the \gls{dds} outputs only the start frequency
of \SI{80}{\mega\hertz} which then is increased by increments to reach the
final frequency of \SI{120}{\mega\hertz} as can be seen in the lower right
spectrogram at the end of the frequency sweep.

For an ideal frequency sweep we would expect a linear timeline of the
frequency, instead we see that \gls{dds} outputs a discrete set of
frequencies.

\subsubsection{Frequency response}

Although the previous section satisfied our curiousity it did not disclose
anything we did not expect. Anyhow, in this part we examine the amplitude
frequency response which by all means has great significance on the observed
intensity patterns we discuss later.

\begin{figure}[h]
  \centering
  \includegraphics[width=\textwidth]{\figuredir{signal/synthesis/frequency-max-amplitude.pdf}}
  \captionsetup{width=.8\textwidth}
  \caption{Maximum amplitude of the two \gls{dds} at different frequencies
  during linear sweep operation. In the two lower curves the \gls{dds} were
configured with enabled inverse sinc filter that is supposed to compensate
frequency dependence in the amplitude.}
  \label{fig:signal_synthesis_frequency_max_amplitude}
\end{figure}

To find the frequency dependency of the amplitude we carried out \gls{fft} on
the voltage of the respective time window and extracted the dominant
frequency located at the maximum of the power spectrum and thereby reduced
each measurement to a frequency and corresponding maximum amplitude.

The result is visualized for the two different signal sources with different
filter mode in \Cref{fig:signal_synthesis_frequency_max_amplitude}. We observe
that there is no significant difference between distinct \gls{dds}. The
inverse sinc filter proclaimed to reduce frequency dependence of the amplitude
only reduced the output amplitude without changing the overall frequency
response characteristics.

\subsection{Summary}

At the signal source stage the \gls{dds} already induce significant frequency
dependency of the output amplitude. The built-in inverse sinc of the \gls{dds}
does not reduce this dependency, therefore we disable the inverse sinc
filter for all other measurements. Eventually we could confirm the
discreteness of the frequency domain produced by the \gls{dds}.

\section{Signal amplification}

The \gls{rf} signal supplied by the synthesizer is too weak to power the
\gls{aod}, therefore we supply the synthesized signal to an amplifier and
feed the amplified signal to the \gls{aod}. The amplification acts as the
second stage in our \gls{rf} signal processing and we expect it to amend
the frequency behaviour of the amplitude.

\begin{figure}[ht]
  \centering
  \includegraphics[width=\textwidth]{\figuredir{signal/amplification/frequency-max-amplitude.pdf}}
  \captionsetup{width=.8\textwidth}
  \caption{Maximum amplitude at dominant frequency per delayed window for two
  different amplifiers. We can see the discreteness of the frequency domain
from the digital signal synthesis as well as three resonances. Further the
amplifier differ by a constant offset.}
  \label{fig:signal_amplification_frequency_max_amplitude}
\end{figure}

\subsection{Setup}

The general setup described in \cref{subsec:signal_synthesis_setup} is still
valid for the now amplified signal, however the amplified signal should now
be of power $\SI{33}{\decibel\meter}\approx\SI{10}{\volt}$ at \SI{50}{\ohm}
which is at the limit of our oscilloscope, therefore it is vital to insert
attentuators in between the amplified \gls{rf} signal and the oscilloscope.

\subsection{Results}

The analysis is analogue but confined to the frequency response part in
\cref{subsec:signal_synthesis_results} as the spectrogram of the amplified
signal would not yield any new insights.

\subsubsection{Frequency response}

\begin{figure}[ht]
  \centering
  \includegraphics[width=\textwidth]{\figuredir{signal/amplification/comparison.pdf}}
  \captionsetup{width=.8\textwidth}
  \caption{Relative amplitude at dominant frequency per delayed window for
  two different amplifiers and their digital signal source. In comparison to
the signal source the amplifiers introduce a further resonance at the
center frequency.}
  \label{fig:signal_amplification_comparison}
\end{figure}

\Cref{fig:signal_amplification_frequency_max_amplitude} presents us the
damped output signal after amplification for the case of our two different
amplifiers. We require two different amplifiers because there are two
independent \gls{aod} to be powered. At first we observe that the amplifiers
differ by a constant offset. We assume this offset to be caused by different
component quality as the frequency response shape itself is similar.

On a second glance we again find evidence for discrete frequencies from
\Cref{fig:signal_amplification_frequency_max_amplitude} when we emphasize the
subtile horizontal pattern of the markers.


For \Cref{fig:signal_amplification_comparison} we compared the normalized
amplified signals with the normalized signal source. We observe similar
signal shape for both amplifiers. Compared to the signal source frequency
response has changed in that there is an additional frequency response in
between of the two responses of the signal source.

\section{Signal reflection}

In the previous section we found.

\begin{figure}[ht]
  \centering
  \includegraphics[width=\textwidth]{\figuredir{signal/reflection/direct.pdf}}
  \captionsetup{width=.8\textwidth}
  \caption{We see the signal reflection of the two different \gls{aod} when
  connected directly to the network analyzer. We can see that both crystal
differ in their respective spectrum.}
  \label{fig:signal_reflection_direct}
\end{figure}

\begin{figure}[ht]
  \centering
  \includegraphics[width=\textwidth]{\figuredir{signal/reflection/coupler.pdf}}
  \captionsetup{width=.8\textwidth}
  \caption{Input power reflection when supplying the direct-coupler with
    \SI{-10}{\decibel\meter} input signal and reflection at the output of
    the direct-coupler while other ports are closed with \SI{50}{\ohm}}.
  \label{fig:signal_reflection_coupler}
\end{figure}

\begin{figure}[ht]
  \centering
  \includegraphics[width=\textwidth]{\figuredir{signal/reflection/coupled.pdf}}
  \captionsetup{width=.8\textwidth}
  \caption{Reflection at the direct-coupler output after amplification of the
  network analyzer input signal for different effective powers. We see that
the applied power does not effect the spectrum.}
  \label{fig:signal_reflection_coupled}
\end{figure}

\begin{figure}[ht]
  \centering
  \includegraphics[width=\textwidth]{\figuredir{signal/reflection/comparison.pdf}}
  \captionsetup{width=.8\textwidth}
  \caption{Comparison of reflection from amplified input signal and direct
  signal provided from the network analyzer. The different reflection spectrum
can be associated to the amplifier.}
  \label{fig:signal_reflection_comparison}
\end{figure}
