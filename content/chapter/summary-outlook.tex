\chapter{Summary and outlook}

We began this thesis with an introduction to the field of ultracold atoms
in optical lattice and motivated the need for local time-averaged optical
potentials as an alternative to harmonic traps, as well as a pathway for new
potential dynamics. From there we dove into the physics of atoms in optical
lattices, in order to obtain an understanding of the requirements placed on
our technical implementation. In this context we found that the deflection of
our optical perurbation needs to operate on a time scale of about
\SI{100}{\kilo\hertz} and that we need a tightly focused laser beam in order
to create high-precision perturbation potentials. Fortunately
Hertlein has already proven in Ref.~\cite{Hertlein2017} that our optical setup
is capable of creating such high-precision beams in the atom plane. So we were
left with exploring the limitations of our electronic and acousto-optic setup.

In \Cref{ch:digital_signal_synthesis} we concluded that digital signal
synthesis is in theory an ideal platform for us to create a stable, phase
continous, high-precision \gls{rf} signal and in fact measurements in
\Cref{ch:electronic_setup} have proven mostly good characteristics of the
used \gls{ad9910} \gls{dds}. However, we also found that our \gls{ad9910}
implementation shows a signal delay when receiving an external trigger of up
to \SI{40}{\micro\second} which is a deal breaker as we need to operate on
time scales below \SI{10}{\micro\second}. Furthermore the frequency resolution
of the digital ramp is too low to create spatially homogenous potentials in
the atom plane. Howsoever we need to recognize that the \gls{ad9910} is sold
as a general purpose \gls{dds}. It is likely that there are other \gls{dds}
that suit our demands better or that one can create an own \gls{dds} design
using \gls{fpga}.

In the second part of our work we investigated in the characteristics of our
\gls{2d} \gls{aod}. We found that the \gls{aod}s exhibit non-linear response
to the relative amplitude and frequency of the applied \gls{rf} signal. It
remains open if its just the \gls{aod} to blame for or if there are electronic
effects we did not unvealed. Evidence that points into the direction that
there may be electronic defects contributing to the non-linear characteristic
of the \gls{aod} where found in \Cref{subsec:different_signal_source} where
we used a signal generator to perform a frequency sweep and found differences
in the diffraction efficiency spectrum. We would like to check if these
observed difference are due to different current characteristics. So far we
only measured the voltage amplitude of the amplified \gls{dds} where we did
not observe significant frequency dependency.

We would also like to dig deeper into the theory of acousto-optics to check
if there are any reports of the observed characteristics and if there are
workarounds. In addition we would find theoretical evidence that our
acousto-optics can handle the targeted time scales. It may be very well the
case that the speed of sound inside the acousto-optics is too slow to deflect
the laser beam on our required time scale. This would also explain the
instability we observed in the attempt to minimize the variance of the
diffraction efficiency. Good literature that should answer these and more
questions can be found in Ref.~\cite{Goutzoulis1994} and Ref.~\cite{Royer1999}.

In case that the \gls{aod} do show theoretial limitations, or we can exclude
electronic defects for the observed characteristics and we need to accept
these characteristics as given, we could also investigate \gls{eod}. \gls{eod}
utilize the electro-optical effect to deflect a laser beam. Usually electrical
fields are well understood and controllable, such that these do not fall under
the same limitations as the \gls{aod}, however \gls{eod} require high voltages
which may be a problem.

Finally there are also interesting industrial applications to consider if the
outstanding issues can be resolved. For example the field of ophthalmologics
covers a wide range of opportunities for precision laser control like
corrective eye surgeries which according to~\cite{US20180110655},
~\cite{US20180064579} and~\cite{US7131968} make use of mechanical deflectors.
Certainly there are also industrial applications in solid-state industries,
yet we did not check up closer.
