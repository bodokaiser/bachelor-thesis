\chapter{Hopping in optical lattices}

In the introduction we presented the concepts behind our enivisaged
generation of local time-averaged high-precision optical potentials. In this
chapter we want to recapitulate the theoretical foundations of optical
lattices and estimate theoretical boundaries for the time-average.

\section{Atom-field interactions}

The Hamiltonian of an electron in an external electromagnetic field reads
\begin{equation}
  \hat{H}_\text{em}
  =\frac{1}{2m}\left(\hat{\vb{p}}+e\vb{A}\right)^2-e\Phi+\hat{V}_0
  \label{eq:hamiltonian_em}
\end{equation}
with vector $\vb{A}$ and scalar potential $\Phi$. The Coulomb potential of
the nucleus resides inside the potential operator $\hat{V}_0$. For an atom
interaction with an external light field we can apply the dipole approximation
$\vb{A}(\vb{x},t)\approx\vb{A}(t)$ \cite{Gerry2004}[p.76].

Gauge freedom permits transformations of type
\begin{align}
  \vb{A}\to\vb{A}+\grad{\chi}
  &&
  \Phi\to\Phi-\pdv{\chi}{t}
\end{align}
with gauge function $\chi$. In the following we choose the gauge function
$\chi=-\vb{A}\cdot\vb{x}$. In the dipole approximation $\chi$ satisfies the
Coulomb gauge condition $\divergence{\vb{A}}=0$ allowing us to set $\Phi=0$
as no external sources are present \cite{Jackson2005}[p.242]. Finally we can
write \cref{eq:hamiltonian_em} as
\begin{equation}
  \hat{H}_\text{dip}
  =\frac{\hat{\vb{p}}^2}{2m}+\hat{V}_0-\hat{\vb{d}}\cdot\vb{E}
  =\frac{\hat{\vb{p}}^2}{2m}+\hat{V}_0+\hat{V}_\text{dip}
  \label{eq:hamiltonian_dip}
\end{equation}
with the dipole operator $\hat{\vb{d}}=-e\vb{x}$ and spatially homogeneous
light field $\vb{E}(t)$.

\subsection{Classical field}

In the quantum treatment of light we would find nonzero probabilities for
transitions between energy levels without the presence of an external light
field, known as spontaneous emissions \cite{Gerry2004}. These dissipative
effects are among others relevant for various laser cooling techniques and
the study of population dynamics where we the laser light is near resonance.
In our use case the classical treatment of light is sufficient as we are
interested in the effective potential seen by the atoms from an
far-off-resonant laser.

\subsubsection{One dimensional lattice}

A one dimensional optical lattices can be generated by creating a standing
wave interference pattern by reflection of a single Gaussian laser beam. The
electrical field component in such a case is given by
\begin{equation}
  \vb{E}(t)
  =\vb{E}_0\cos(kz)\cos(\omega t)\exp(-\frac{2r^2}{w(z)^2})
  \label{eq:1d_lattice_efield_exact},
\end{equation}
wherein $r$ denotes the beam radius, $w(z)$ being spot size parameter and
$k=2\pi/\lambda$ with the laser wavelength $\lambda$. Analogue to
\cite[p.127]{Rom2009} we assume $w(z)\gg r$ and drop the exponential,
simplifying \cref{eq:1d_lattice_efield_exact} to
\begin{equation}
  \vb{E}(t)
  \approx\vb{E}_0\cos(kz)\cos(\omega t)
  \label{eq:1d_lattice_efield_approx}.
\end{equation}
Note that the dipole approximation still holds for
\cref{eq:1d_lattice_efield_approx} as for each atom the $z$ coordinate
will be approximately constant.

\subsubsection{Second order perturbation}

In the case of potential generation for ultracold atoms, we want to avoid any
population dynamics and keep the atoms in their respective ground state.
Therefore our optical potential is created with a far-off-resonance laser
frequency. Second order perturbation theory predicts an energy shift of
\begin{equation}
  \Delta E_n
  \approx
  \mel{\psi_n}{\hat{V}_\text{dip}}{\psi_n}
  +
  \sum_{m\neq n}\frac{\abs{\mel{\psi_m}{\hat{V}_\text{dip}}{\psi_n}}^2}{E_n-E_m}
  \label{eq:second_order_perturbation}.
\end{equation}
Symmetry of the atomic wave functions only allows nonzero dipole moments for
state transitions, therefore $\mel{\psi_n}{\hat{V}_\text{dip}}{\psi_n}=0$.
Further we assume only the first two energy levels to be relevant for
ultracold atoms, yielding us a two-level system and simplifying
\cref{eq:second_order_perturbation} to
\begin{equation}
  \Delta E
  \approx
  \pm\frac{\abs{\mel{\psi_0}{\hat{V}_\text{dip}}{\psi_1}}^2}{\abs{E_0-E_1}}
  =
  \pm\frac{c\epsilon_0 \abs{d_{01}}^2}{2\hbar\delta}I_0\cos^2(kz)
  \label{eq:light_shift}
\end{equation}
with detuning $\delta=\abs{\omega_{01}-\omega}$, maximum laser intensity
$I_0$, dielectric constant $\epsilon_0$, vacuum speed of light $c$ and dipole
moment $d_{01}$ for the state transition from ground to first excited state
with linear polarized light. The sign in \cref{eq:light_shift} has to be
chosen appropriate to the direction of the laser detuning. For red detuned
light the negative sign has to be choosen and atoms will be confined to the
nodes of the standing light wave. Explicit values for general dipole
transition elements for one and two electron systems can be found in
\cite{Bethe1957}.

For the upcoming sections we will take the effective potential in the reference
frame of atoms in an one dimensional optical lattice to be
\begin{equation}
  V_\text{lat}(z)=V_0\cos^2(kz)
  \label{eq:potential_lattice}
\end{equation}
with $V_0<0$ for red detuned light. The wave number $k$ of the laser source
is connected to the lattice constant of the optical lattice by
$a=\frac{\pi}{k}$.

\section{Lattice band structure}

With the previous foundations in place we can proceed with the solid state
analogues for optical lattices. From the  the simple analytic form of the
optical potential, \cref{eq:optical_lattice_potential}, we are able to
derive an expression for the quantum tunneling probability which gives us
an upper bound for the duration time of our potential generation.

\subsection{Bravais lattice}

A Bravais lattice of dimension $N$ consists of the discrete points
\begin{equation}
  \vb{r}
  =\bra{\vb{x}}\ket{\vb{r}}
  =\sum_{i=1}^Nn_i\vb{a}_i
  \qc n_1,\dots,n_N\in\mathbb{Z}
\end{equation}
where the $\vb{a}_i$ are the primitive vectors that span the primitive cell.
The reciprocal lattice is defined as the point set
\begin{equation}
  \vb{g}
  =\bra{\vb{x}}\ket{\vb{g}}
  =\sum_{i=1}^Nm_i\vb{b}_i
  \qc m_1,\dots,m_N\in\mathbb{Z}
\end{equation}
that satisfy the condition $\exp(\vb{r}\cdot\vb{g})=1$. For a simple cubic
lattice structure $\vb{b}_i=2\pi\vb{a}/{{\vb{a}_i}^2}$ satisfies said
condition.

In the reciprocal lattice space it is especially simple to express periodic
functions. This suggests the transformation of a periodic potential
\cref{eq:potential_lattice} into the reciprocal lattice space.
\begin{equation}
  \bra{g_n}\hat{V}\ket{g_m}
  &=\int\dd{x}\bra{g_n}\ket{x}V(x)\overline{\bra{g_m}\ket{x}}
  \label{eq:transform_reciprocal_lattice}
\end{equation}
With $\bra{x}\ket{g_m}=e^{2ikxm}/\sqrt{2\pi}$ we can identify
\cref{eq:transform_reciprocal_lattice} as Fourier transform into the
reciprocal lattice base. Taking \cref{eq:potential_lattice} as periodic
potential and using orthogonality between Fourier modes we find
\begin{equation}
  \bra{g_n}\hat{V}_\text{lat}\ket{g_m}
  =\frac{1}{4}V_0\left(\delta_{n,m-1}+\delta_{n,m+1}+2\delta_{n,m}\right)
  \label{eq:potential_reciprocal_lattice_base}.
\end{equation}

\subsection{Bloch functions}

Bloch's theorem states that for a periodic potential
\begin{equation}
  V_\text{per}(\vb{x}+\vb{a})=V_\text{per}(\vb{x})
  \label{eq:potential_periodic}
\end{equation}
there exists a complete set of wavefunctions that are energy eigenstates
of the Hamiltonian
\begin{equation}
  \hat{H}_\text{per}
  =\frac{\hat{\vb{p}}^2}{2m}+\hat{V}_\text{per}
  \label{eq:hamiltonian_periodic},
\end{equation}
and each of these Bloch waves can be written in the form
\begin{equation}
  \bra{\vb{x}}\ket{\Psi_{n\vb{q}}}
  =\Psi_{n\vb{q}}(\vb{x})
  =e^{i\vb{q}\cdot\vb{x}}\psi_{n\vb{q}}(\vb{x})
  \label{eq:wave_bloch}
\end{equation}
with $\psi_{n\vb{q}}(\vb{x}+\vb{a})=\psi_{n\vb{q}}(\vb{x})$, wave vector
$\vb{q}$ and bandindex $n$. For a proof of Bloch's theorem see
\cite{Roessler2004} or \cite{Bartelmann2018}. We confine the wave number to
the first Brillouin zone $-k<q<+k$ to obtain an unique set of Bloch waves.

\subsection{Energy bands}

Using the Bloch waves \cref{eq:wave_bloch} as ansatz for the Hamiltonian
\begin{equation}
  \hat{H}_\text{lat}
  =\frac{\hat{\vb{p}}}{2m}+\hat{V}_\text{lat}
  \label{eq:hamiltonian_lattice},
\end{equation}
$V_\text{lat}(z)=V_0\cos^2(kz)$ found in \cref{eq:potential_periodic}, we are
now able to find an expression for the energy eigenvalues. We notice that by
the product rule
$\hat{p}^2\ket{\Psi_{nq}}=e^{iqx}(\hat{p}+\hbar q)\ket{\psi_{nq}}$,
thus finding
\begin{equation}
  E_n(q)\ket{\Psi_{nq}}
  =\hat{H}_{lat}\ket{\Psi_{nq}}
  =\left(\frac{(\hat{p}+\hbar q)^2}{2m}+\hat{V}_\text{lat}\right)\ket{\psi_{nq}}
  \label{eq:eigenvalue_energy_lattice}.
\end{equation}

We expand the state $\ket{\psi_{nq}}$ in the reciprocal lattice base
\begin{equation}
  \ket{\psi_{nq}}
  =\left(\sum_j\ket{g_j}\bra{g_j}\right)\ket{\psi_{nq}}
  =\sum_j\braket{g_j}{\psi_{nq}}\ket{g_j}
  =\sum_jc_j\ket{g_j}
  \label{eq:state_expansion_reciprocal_lattice_base}.
\end{equation}
We need to find the momentum eigenvalues of the state $\ket{g_j}$
\begin{equation}
  \hat{p}\ket{g_j}
  =\int\int\dd{x}\dd{y}\ket{y}\bra{y}\hat{p}\ket{x}\bra{x}\ket{g_j}
  \label{eq:eigenvalue_momentum_reciprocal_lattice1}.
\end{equation}
With $\bra{y}\hat{p}\ket{x}=-i\hbar\delta(y-x)\dv{x}$ we can take the
derivative of $\bra{x}\ket{g_j}=e^{2ikxj}$ and simplify
\cref{eq:eigenvalue_momentum_reciprocal_lattice1} to
\begin{equation}
  \hat{p}\ket{g_j}
  =2\hbar kj\int\dd{x}\ket{x}\bra{x}\ket{g_j}
  =2\hbar kj\ket{g_j}
  \label{eq:eigenvalue_momentum_reciprocal_lattice1}.
\end{equation}

Finally we insert \cref{eq:state_expansion_reciprocal_lattice_base} into
\cref{eq:eigenvalue_energy_lattice} and apply $\bra{g_i}$ to the right hand
side while using \cref{eq:potential_reciprocal_lattice_base} and
\cref{eq:eigenvalue_momentum_reciprocal_lattice1}, yielding
\begin{equation}
  E_n(q)
  =\bra{g_i}\hat{H}_\text{lat}\ket{g_j}
  =
  \begin{cases}
    (j+q/k)^2E_r+\frac{1}{2}V_0, & \text{if } i=j\\
    \frac{1}{4}V_0, & \text{if } \abs{i-j}=1\\
    0, & \text{otherwise}
  \end{cases}
  \label{eq:eigenvalue_energy_energy_lattice_final}
\end{equation}
wherein we introduced the recoil energy $E_r=\hbar^2k^2/(2m)$ which gives us
a natural, atom independent energy scale.

\subsection{Harmonic approximation}

Another approach to solve \cref{eq:hamiltonian_lattice} in closed form is to
use the fact that cooled atoms have a low kinetic energy and will be located
at the bottom of the potential wells $\hat{V}_\text{lat}$ such that we can
do an harmonic approximation
\begin{equation}
  V_\text{lat}(x)
  =V_0\cos^2(kx)
  \approx V_0\left(1-(kx)^2\right)
  \label{eq:potential_lattice_approximation}
\end{equation}
inserting \cref{eq:potential_lattice_approximation} into
\cref{eq:hamiltonian_lattice} and rearranging yield us the Hamiltonian of
the harmonic oscillator
\begin{equation}
  \frac{\hbar\omega}{2}\left(2n+1\right)\ket{\phi}
  =\left(E_n-V_0\right)\ket{\phi}
  =\left(\frac{\hat{p}^2}{2m}-V_0k^2x^2\right)\ket{\phi}
  =\hat{H}_\text{har}\ket{\phi}
\end{equation}
with well-known energy levels and frequency
$\omega=\sqrt{V_0/m}k=\sqrt{2V_0E_r}/\hbar$. In the harmonic approximation
the energy bands found in \cref{eq:eigenvalue_energy_energy_lattice_final}
decrease to single energy levels. By contrast flat potentials would increase
the width of the energy bands which in the limit leads to free atoms. These
cases are known in solid state physics as tight-binding approximation and
nearly-free electron model.

\section{Hopping between lattice sites}

\subsection{Wannier functions}

\subsection{Probability amplitude}
