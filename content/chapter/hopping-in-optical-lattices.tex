\chapter{Hopping in optical lattices}

In the introduction we presented the concept of our envisaged potential
creation. In the following chapter we want to recapitulate the theoretical
foundations of optical lattices and quantum states theoreof. Finally we want
to derive an expression for the quantum tunnelling frequency in order to
estimate the time magnitude in which our apparatus has to operate.

\section{Atom-light interaction}

At first we need to ask ourselves how the laser field propagates as
perturbation potential to the Hamiltonian of the atomic system and how we can
express said potential by physical quantities we control in the experiment.

\subsection{Hamiltonian and dipole approximation}

The Hamiltonian of an electron in an external electromagnetic field reads
\begin{equation}
  \hat{H}_\text{em}
  =\frac{1}{2m}\left(\hat{\vb{p}}+e\vb{A}\right)^2-e\Phi+\hat{V}_0
  \label{eq:hamiltonian_em}
\end{equation}
with vector $\vb{A}$ and scalar potential $\Phi$ of the external field and
the Coulomb potential of the nucleus $\hat{V}_0$. \Cref{eq:hamiltonian_em} is
exact for the hydrogen atom and approximate for alkali atoms with one outer
electron.

Gauge freedom of the vector and scalar potential permits transformations of
type
\begin{align}
  \vb{A}\to\vb{A}+\grad{\chi}
  &&
  \Phi\to\Phi-\pdv{\chi}{t}
\end{align}
with gauge function $\chi(\vb{x},t)$. In the following we choose the gauge
function $\chi=-\vb{A}\cdot\vb{x}$ and assume the dipole approximation
$\vb{A}(\vb{x},t)\approx\vb{A}(t)$. The dipole approximation is reasonable as
wave length of visible light are much larger then atomic length scales
\cite{Gerry2004}. In the dipole approximation $\chi$ satisfies the Coulomb
gauge condition $\divergence{\vb{A}}=0$ allowing us to set $\Phi=0$ as no
external sources are present \cite{Jackson2005}. Finally we can rewrite
\cref{eq:hamiltonian_em} as
\begin{equation}
  \hat{H}_\text{dip}
  =\frac{\hat{\vb{p}}^2}{2m}+\hat{V}_0-\hat{\vb{d}}\cdot\vb{E}
  =\hat{H}_0+\hat{V}_\text{dip}
  \label{eq:hamiltonian_dip}
\end{equation}
with the dipole operator $\hat{\vb{d}}=-e\vb{x}$ and spatially homogeneous
light field $\vb{E}(t)$.

\subsection{Field of an optical lattice}

An one dimensional optical lattices can be generated through the interference
of two counter-propagating Gaussian beams, creating a standing wave
interference pattern. The electrical field component in such a case is given
by
\begin{equation}
  \vb{E}(t)
  =\vb{E}_0\cos(kz)\cos(\omega t)\exp(-\frac{2r^2}{w(z)^2})
  \label{eq:1d_lattice_efield_exact},
\end{equation}
wherein $r$ denotes the beam radius, $w(z)$ the spot size parameter and $k$
the angular wavenumber of the laser. The wavenumber relates to the laser
wavelength via $k=2\pi/\lambda$. The magnetic field component is chosen
in such a way that it satisfies the Maxwell equations but will not be
considered relevant any further as the electron only has a small magnetic
dipole. Analogue to \cite{Rom2009} we will now assume $w(z)\gg r$ and drop
the exponential term in \cref{eq:1d_lattice_efield_exact}, thus simplifying
\cref{eq:1d_lattice_efield_exact} to
\begin{equation}
  \vb{E}(t)
  \approx\vb{E}_0\cos(kz)\cos(\omega t)
  \label{eq:1d_lattice_efield_approx}.
\end{equation}
Note that even the electric field in \cref{eq:1d_lattice_efield_approx} has
a spatial dependency, the dipole approximation still holds in the atomic
reference frame.

\subsection{Effective dipole potential}

We are now going to solve \cref{eq:hamiltonian_dip} for the case of an
electric field of the form \cref{eq:1d_lattice_efield_approx} with
for-off-resonant laser wavelength $\lambda$ and finally derive an expression
for the effective dipole potential.

At $t<0$ the system is in the energy eigenstate $\ket{n}$ of the unperturbated
Hamiltonian
\begin{equation}
  \hat{H}_0\ket{n}
  =E_n\ket{n}
  =\hbar\omega_n\ket{n}
  \label{eq:eigenvalue_energy_unperturbated}.
\end{equation}
At $t>0$ the external light field appears instantaneous. The new state
$\ket{\psi}$ can be expanded in the complete base of the previous energy
eigenstates
\begin{equation}
  \ket{\psi}
  =\sum_nc_n(t)e^{-i\omega_nt}\ket{n}
  \label{eq:state_expansion_unperturbated}.
\end{equation}
Inserting \cref{eq:state_expansion_unperturbated} into the time-dependent
Schrödinger equation with dipole Hamiltonian \cref{eq:hamiltonian_dip} and
applying $\bra{m}e^{i\omega_mt}$ to the right hand side leads us to a set
of differential equations
\begin{equation}
  \dot{c}_m
  =-\frac{i}{\hbar}\sum_nc_n(t)e^{-i\omega_{nm}t}\bra{m}\hat{V}_\text{dip}\ket{n}
  \label{eq:differential_equation_population_dynamics}
\end{equation}
with $\omega_{nm}=\omega_n-\omega_m$, that describe the dynamics of the
probablity amplitudes $c_n(t)$. By using the electric field we derived in
\cref{eq:1d_lattice_efield_approx} we can rewrite the dipole transition
matrix elements
\begin{equation}
  \bra{m}\hat{V}_\text{dip}\ket{n}
  =\hbar\Omega_{nm}\cos(kz)\cos(\omega t)
  \label{eq:elements_dipole_transition_matrix}
\end{equation}
where we introduced the Rabi frequency
$\Omega_{nm}=\bra{n}\hat{\vb{d}}\cdot\vb{E}_0\ket{m}/\hbar$. Explicit values
for general dipole transition elements for one and two electron systems can
be found in \cite{Bethe1957}. Als the transition elements vanishes for states
with same parity, in particular do we have $\Omega_{nn}=0$
\cite{Bartelmann2018}.

From now on we assume a two state system that initially is only populated in
the ground state $c_g(0)=1,c_e(0)=0$ and the dynamics in
\cref{eq:differential_equation_population_dynamics} are simply described by
\begin{align}
  i\dot{c}_g=\Omega_0c_e(t)\cos(kz)\cos(\omega t)e^{+i\omega_0 t} &&
  i\dot{c}_e=\Omega_0c_g(t)\cos(kz)\cos(\omega t)e^{-i\omega_0 t}
  \label{eq:differential_equation_population_dynamics_two_state_system}.
\end{align}
Expansion of $\cos(\omega t)$ in terms of the exponential function and
dropping $e^{\pm i(\omega+\omega_0)t}$ yields
\begin{align}
  i\dot{c}_g\approx\frac{\Omega_0}{2}c_e(t)\cos(kz)e^{+i\Delta\omega t} &&
  i\dot{c}_e\approx\frac{\Omega_0}{2}c_g(t)\cos(kz)e^{-i\Delta\omega t}
  \label{eq:differential_equation_population_dynamics_two_state_system_rwa}.
\end{align}
The so-called rotating wave approximation is motivated by the fact that
oscillations of frequency $\omega+\omega_0$ are fast compared to changes in
the population dynamics and therefore vanish on average.

We now define $a_g=c_g$ and $a_e=c_ee^{i\Delta\omega t}$ and refine
\cref{eq:differential_equation_population_dynamics_two_state_system_rwa} by
\begin{align}
  i\dot{a}_g=\frac{\Omega_0}{2}a_e(t)\cos(kz) &&
  i\dot{a}_e=\frac{\Omega_0}{2}a_g(t)\cos(kz)-a_e\Delta\omega
  \label{eq:differential_equation_population_dynamics_two_state_system_shift}.
\end{align}
In the former form we can diagonalize the Hamiltonian and find energy
eigenvalues to be
\begin{equation}
  E_{e,g}
  =\frac{\hbar}{2}\left(-\Delta\omega\mp\sqrt{\Omega_0^2+\Delta\omega^2}\right)
  \approx
  =\mp\frac{\hbar\Omega_0^2}{4\Delta\omega}
  \label{eq:eigenvalues_energy_light_shift}
\end{equation}
where we applied Taylor expansion to the square root for
$\Delta\omega\gg\Omega_0$.

In summary atoms in an external off-resonant light field with experience an
effective periodic dipole potential
\begin{equation}
  \hat{V}_\text{eff}
  =V_0\cos^2(kz)
  \qc
  V_0
  =\frac{\hbar\Omega}{4\Delta\omega}\cos^2(kz)
  =\frac{\hbar I_0\cos^2(kz)}{2c_0\epsilon_0\Delta\omega}
  \label{eq:potential_effective}
\end{equation}
with vacuum speed of light $c_0$, dielectric constant $\epsilon_0$ and
maximum intensity $I_0$. Same results can also be obtained by the use of
second order perturbation theory, however the presented approach is more
clear on assumptions and time dependence \cite{Grimm2008}.

\section{Harmonic approximation}

Given the effective lattice potential \cref{eq:potential_effective} the
effective lattice Hamiltonian reads
\begin{equation}
  \hat{H}_\text{eff}
  =\frac{\hat{p}^2}{2m}+\hat{V}_\text{eff}
  \label{eq:hamiltonian_effective}.
\end{equation}
One first naive approach to solve the time-independent Schrödinger equation
subject to the effective lattice Hamiltonian would be to Taylor expand
the effective potential in second order
\begin{equation}
  \hat{V}_\text{eff}
  =V_0\cos^2(kx)
  \approx V_0\left(1-(kx)^2\right)
  \label{eq:potential_harmonic_approximation}
\end{equation}
and give us the Hamiltonian of a linear harmonic quantum oscillator
\begin{equation}
\begin{equation}
  \hat{H}_\text{har}-V_0
  =\frac{\hat{p}^2}{2m}-V_0k^2x^2
  \label{eq:hamiltonian_harmonic_approximation}
\end{equation}
with well-known energy levels $E_n=\hbar\omega(2n+1)/2$ and frequency
$\omega=\sqrt{V_0/m}k=\sqrt{2V_0E_r}/\hbar$, where we defined the recoil
energy $E_r=(\hbar k)^2/(2m)$ as an atom-independent energy scale.

The harmonic approximation is valid for very deep potentials $V_0\gg E$.
Unfortunately the Gaussian wave functions of the harmonic oscillator decay
fast at the potential boundary such that we have vanishing probability to
find a particle tunneling through the potential.

\section{Periodic lattice potentials}

The following section is dedicated to the quantum mechanics of periodic
lattice potentials. After we revise the formulation of lattice structures,
we will examine Bloch and Wannier states as well as the energy bands arising
in periodic lattice potentials.

\subsection{Lattice structures}

A Bravais lattice of dimension $N$ consists of the discrete points
\begin{equation}
  \vb{r}
  =\bra{\vb{x}}\ket{\vb{r}}
  =\sum_{i=1}^Nn_i\vb{a}_i
  \qc n_1,\dots,n_N\in\mathbb{Z}
\end{equation}
where $\vb{a}_i$ are the primitive vectors that span the primitive lattice
cell.

We define the reciprocal lattice as the discrete points
\begin{equation}
  \vb{g}
  =\bra{\vb{p}}\ket{\vb{g}}
  =\sum_{i=1}^Nm_i\vb{b}_i
  \qc m_1,\dots,m_N\in\mathbb{Z}
\end{equation}
that satisfy the condition $\exp(\vb{r}\cdot\vb{g})=1$. For a simple cubic
lattice structure $\vb{b}_i=2\pi\vb{a}/{{\vb{a}_i}^2}$ satisfies said
condition. For the subsequent sections we will only consider the one
dimensional simple cubic lattice. Cubic lattices of higher dimension however
can be easily obtained by superposition of the potentials.

As the reciprocal lattice space can be seen as the Fourier transform of the
Bravais lattice, it is especially simple to express periodic functions and it
suggests the representation of periodic potentials in reciprocal lattice
space. Let $\ket{g_n}$ denote the $n$th reciprocal lattice point then the
transformation rule is
\begin{equation}
  \bra{g_n}\hat{V}\ket{g_m}
  =\int\dd{x}\bra{g_n}\ket{x}^*V(x)\bra{g_m}\ket{x}
  =\int\dd{x}V(x)e^{2ikx(n-m)}
  \label{eq:potential_in_reciprocal_lattice}.
\end{equation}
Evaluation of \cref{eq:potential_in_reciprocal_lattice} for a specific
potential will usually yield a finite number of terms in dependence of
$n-m$ as we actually determine the Fourier coefficients of a discrete Fourier
transform.

\subsection{Bloch states}

Bloch's theorem states that for a periodic lattice potential
\begin{equation}
  V_\text{lat}(\vb{x}+\vb{a})=V_\text{lat}(\vb{x})
  \label{eq:potential_periodic}
\end{equation}
there exists a complete set of wavefunctions that are energy eigenstates
of the Hamiltonian
\begin{equation}
  \hat{H}_\text{lat}
  =\frac{\hat{\vb{p}}^2}{2m}+\hat{V}_\text{lat}
  \label{eq:hamiltonian_periodic},
\end{equation}
and each of these Bloch waves can be written into the form
\begin{equation}
  \bra{\vb{x}}\ket{\Psi^n_{\vb{q}}}
  =\Psi^n_{\vb{q}}(\vb{x})
  =e^{i\vb{q}\cdot\vb{x}}\psi^n_{\vb{q}}(\vb{x})
  \label{eq:state_bloch}
\end{equation}
with $\psi^n_{\vb{q}}(\vb{x}+\vb{a})=\psi^n_{\vb{q}}(\vb{x})$, wave vector
$\vb{q}$ and bandindex $n$. We confine the wave number to the first Brillouin
zone $-k<q<+k$. For a proof of Bloch's theorem see \cite{Roessler2004} or
\cite{Bartelmann2018}.

\subsection{Energy bands}

The Bloch states \cref{eq:state_bloch} can be used as ansatz to solve the
periodic lattice Hamiltonian
\begin{equation}
  \hat{H}_\text{lat}
  =\frac{\hat{p}^2}{2m}+\hat{V}_\text{lat}
  \label{eq:hamiltonian_lattice}.
\end{equation}
We notice that by the product rule
$\hat{p}^2\Psi^n_q(x)=e^{iqx}(\hat{p}+\hbar q)\psi^n_q(x)$,
thus we find
\begin{equation}
  E^n_q\ket{\Psi^n_{q}}
  =\hat{H}_{lat}\ket{\Psi^n_{q}}
  =e^{iqx}\left(\frac{(\hat{p}+\hbar q)^2}{2m}+\hat{V}_\text{lat}\right)\ket{\psi^n_{q}}
  \label{eq:eigenvalue_energy_lattice}.
\end{equation}
Expansion of the state $\ket{n,q}$ in states of the reciprocal lattice returns
\begin{equation}
  \ket{\Psi^n_q}
  =\left(\sum_s\ket{g_s}\bra{g_s}\right)\ket{\Psi^n_q}
  =\sum_s\braket{g_s}{\Psi^n_q}\ket{g_s}
  =\sum_sc^n_{sq}\ket{g_s}
  \label{eq:state_bloch_in_reciprocal_lattice}.
\end{equation}
The momentum eigenvalues of the state $\ket{g_s}$ can be found by expansion
into position space
\begin{equation}
  \hat{p}\ket{g_s}
  =\int\int\dd{x}\dd{y}\ket{y}\bra{y}\hat{p}\ket{x}\bra{x}\ket{g_s}
  \label{eq:eigenvalue_momentum_reciprocal_lattice1}.
\end{equation}
With $\bra{y}\hat{p}\ket{x}=-i\hbar\delta(y-x)\dv{x}$ we can take the
derivative of $\bra{x}\ket{g_s}=e^{2ikxs}$ and simplify
\cref{eq:eigenvalue_momentum_reciprocal_lattice1} down to
\begin{equation}
  \hat{p}\ket{g_s}
  =2\hbar ks\int\dd{x}\ket{x}\bra{x}\ket{g_s}
  =2\hbar ks\ket{g_s}
  \label{eq:eigenvalue_momentum_reciprocal_lattice2}.
\end{equation}
Finally we insert \cref{eq:state_bloch_in_reciprocal_lattice} into
\cref{eq:eigenvalue_energy_lattice} and apply $\bra{g_t}$ to the
right hand side while using \cref{eq:potential_in_reciprocal_lattice} and
\cref{eq:eigenvalue_momentum_reciprocal_lattice2}, yielding
\begin{equation}
  E^n_qc^n_{tq}
  =c^n_{tq}\frac{(2s+q/k)^2}{2m}E_r+\sum_sc^n_{sq}\bra{g_t}\hat{V}_\text{lat}\ket{g_s}
  \label{eq:eigenvalue_energy_lattice_explicit}
\end{equation}
with recoil energy $E_r=\hbar^2k^2/(2m)$.
\Cref{eq:eigenvalue_energy_lattice_explicit} has to be solved for known
$\hat{V}_\text{lat}$ numerically. We will do this later for the effective
optical lattice potential from \cref{eq:potential_effective}.

\subsection{Wannier states}

So far we studied the effects of a periodic lattice potential on the energy
levels and the Bloch wave functions that propagate over the complete lattice.
Lattice hopping however is a local effect, therefore we need a spatially
localized set of wave functions to describe the effect of lattice hopping.
Fortunately the so called Wannier fullfill these requirements.

The orthodox definition of the Wannier function with bandindex $n$ at lattice
site $l$ reads
\begin{equation}
  \ket{\phi^n_{l}}
  =\frac{1}{\sqrt{N}}\sum_qe^{-iqla}\ket{\Psi^n_{q}}
  \label{eq:state_wannier_orthodox}
\end{equation}
wherein $\ket{\Psi^n_q}$ is the Bloch wave functions from
\cref{eq:state_bloch}, $N$ the total number of lattice sites and the sum
over $q$ is confined to the first Brillouin zone with $q$ that satisfy the
periodic boundary condition $-k<q=2ki/N<+k$ with integer $i\in\mathbb{Z}$.

The defintion \cref{eq:state_wannier_orthodox} is valid for simple lattice
structures, however for more sophisticated lattices Gauge freedom of the
phase of the Bloch states leads to different Wannier states, whereof only
one is the maximally localized one \cite{Goerg2014}. One can find the
phase of the Bloch state that is maximally localized by minimizing the spread
of the Wannier states, however there also exists a more general definition
that incorporates the phase ubiquity. In the general definition the Wannier
states are best to construct as the eigenstates of the band projected
position operator. The band projector to bandindex $n$ is defined as
\begin{equation}
  \hat{P}_n
  =\sum_q\ket{\Psi^n_q}\bra{\Psi^n_q}
  \label{eq:operator_band_projector}
\end{equation}
and the band projected position operator just follows as
\begin{equation}
  \hat{x}_n
  =\hat{P}_n\hat{x}_n\hat{P}_n
  \label{eq:operator_band_projected_position}.
\end{equation}
The wannier states are now defined as the eigenstates of the eigenvalue
equation
\begin{equation}
  \hat{x}_n\ket{\phi^n_l}=x^n_l\ket{\phi^n_l}
  \label{eq:eigenvalue_band_projected_position}
\end{equation}
where $x^n_l$ is the $l$th lattice site in the $n$th energy band. Choosing
the Bloch base to find the elements of the band projected position operator
\cref{eq:operator_band_projected_position} through
\begin{equation}
  X^{nn^\prime}_{qq^\prime}
  =\bra{\Psi^n_q}\hat{x}_n\ket{\Psi^{n^\prime}_{q^\prime}}
  =\int_0^{Na}\dd{x}\Psi^n_q(x)^*x\Psi^{n^\prime}_{q^\prime}(x)
  \label{eq:element_band_projected_position}
\end{equation}
which give an analytical expression to solve to determine the matrix elements
\cref{Bissbort2013}. Let $d^{mn}_{lq}$ be the diagonalized
matrix $X^{nn^\prime}_{qq^\prime}$ in
\cref{eq:element_band_projected_position} then the Wannier states are fully
determined by
\begin{equation}
  \ket{\phi^n_l}
  =\sum_{n,q}d^{mn}_{lq}\ket{\Phi^n_q}
  \label{eq:state_wannier}.
\end{equation}

\section{Hopping amplitude}

With the previous tools in place we are now set to give an expression for the
quantum hopping amplitude between lattices.

We define the hopping probability from lattice site $l$ to lattice site
$l^\prime$ at bandindex $n$ as
\begin{equation}
  J^n_{ll^\prime}
  =-\bra{\phi^n_l}\hat{H}_\text{lat}\ket{\phi^n_{l^\prime}}
  \label{eq:element_hopping}
\end{equation}
this definition makes intuitive sense if we remind us at the dipole transition
element and the Wannier states. Using the orthodox definition of the Wannier
states \cref{eq:state_wannier_orthodox} for \cref{eq:element_hopping} we find
\begin{equation}
  J^n_{ll^\prime}
  =\frac{1}{N}\sum_{q,q^\prime}e^{iqa(l-l^\prime)}
  \bra{\Psi^n_q}\hat{H}_\text{lat}\ket{\Psi^n_{q^\prime}}
  \label{eq:element_hopping2}.
\end{equation}
The Bloch states are energy eigenstates
\cref{eq:eigenvalue_energy_lattice_explicit} and orthonormal, henceforth
\begin{equation}
  J^n_{ll^\prime}
  =\frac{1}{N}\sum_{q}E^n_qe^{iqa(l-l^\prime)}
  \label{eq:element_hopping3}.
\end{equation}
Expression \cref{eq:element_hopping3} is exact for a given number of lattices
$N$ and can be evaluated numerically. The form of \cref{eq:element_hopping3}
resembles a discrete Fourier transform. The inverse Fourier transform gives us
the energies in terms of the hopping probabilites
\begin{equation}
  E^n_q
  =\sum_{l-l^\prime}J^n_{l,l^\prime}e^{-iqa(l-l^\prime)}
  \label{eq:element_energy_hopping}
\end{equation}
which in practice is easier to calculate then \cref{eq:element_hopping} as
can be seen in the consecutive section.

\subsection{Tight-binding approximation}

For deep potentials $V_0\gtrapprox3E_r$ hopping only contributes from direct
neighbour sites \cite{Rom2009}. Thus we can \cref{eq:element_energy_hopping}
reduces to
\begin{equation}
  E^n_q
  \approxJ^n_{l,l+1}e^{-iqa}+J^n_{l,l-1}e^{+iqa}+J^n_{ll}
  =J^n_0+2J^n_1\cos(qa)
  \label{eq:element_energy_hopping_tb_approx}
\end{equation}
where we have $J^n_{l,l-1}=J^n_{l,l+1}=J_1$ because of translation invariance.
Using evaluation of $J^n_0=J^n_{ll}$ with \cref{eq:element_hopping} tells us
that $J^n_0$ just is the average energy of an energy band. Evaluation of
\cref{eq:element_energy_hopping_tb_approx} at $q=0$ and $q=k$ and subtracting
the results from another yields us
\begin{equation}
  J^n_1\approx\frac{1}{4}\left(E^n_k-E^n_0\right)
  \label{eq:hopping_amplitude_energies}
\end{equation}
which says that the hopping probability is approximate one quarter of the
energy bandwidth.
