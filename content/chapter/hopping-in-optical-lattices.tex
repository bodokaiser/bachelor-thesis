\chapter{Hopping in optical lattices}

In the introduction we presented the concept of our envisaged potential
creation. In the following chapter we want to recapitulate the theoretical
foundations of optical lattices and quantum states theoreof. Finally we want
to derive an expression for the quantum tunnelling frequency in order to
estimate the time magnitude in which our apparatus has to operate.

\section{Atom-light interaction}

At first we need to ask ourselves how the laser field propagates as
perturbation potential to the Hamiltonian of the atomic system and how we can
express said potential by physical quantities we control in the experiment.

\subsection{Hamiltonian and dipole approximation}

The Hamiltonian of an electron in an external electromagnetic field reads
\begin{equation}
  \hat{H}_\text{em}
  =\frac{1}{2m}\left(\hat{\vb{p}}+e\vb{A}\right)^2-e\Phi+\hat{V}_0
  \label{eq:hamiltonian_em}
\end{equation}
with vector $\vb{A}$ and scalar potential $\Phi$ of the external field and
the Coulomb potential of the nucleus $\hat{V}_0$. \Cref{eq:hamiltonian_em} is
exact for the hydrogen atom and approximate for alkali atoms with one outer
electron.

Gauge freedom of the vector and scalar potential permits transformations of
type
\begin{align}
  \vb{A}\to\vb{A}+\grad{\chi}
  &&
  \Phi\to\Phi-\pdv{\chi}{t}
\end{align}
with gauge function $\chi(\vb{x},t)$. In the following we choose the gauge
function $\chi=-\vb{A}\cdot\vb{x}$ and assume the dipole approximation
$\vb{A}(\vb{x},t)\approx\vb{A}(t)$. The dipole approximation is reasonable as
wave length of visible light are much larger then atomic length scales
\cite{Gerry2004}. In the dipole approximation $\chi$ satisfies the Coulomb
gauge condition $\divergence{\vb{A}}=0$ allowing us to set $\Phi=0$ as no
external sources are present \cite{Jackson2005}. Finally we can rewrite
\cref{eq:hamiltonian_em} as
\begin{equation}
  \hat{H}_\text{dip}
  =\frac{\hat{\vb{p}}^2}{2m}+\hat{V}_0-\hat{\vb{d}}\cdot\vb{E}
  =\hat{H}_0+\hat{V}_\text{dip}
  \label{eq:hamiltonian_dip}
\end{equation}
with the dipole operator $\hat{\vb{d}}=-e\vb{x}$ and spatially homogeneous
light field $\vb{E}(t)$.

\subsection{Electrical field of an 1D optical lattice}

An one dimensional optical lattices can be generated through the interference
of two counter-propagating Gaussian beams, creating a standing wave
interference pattern. The electrical field component in such a case is given
by
\begin{equation}
  \vb{E}(t)
  =\vb{E}_0\cos(kz)\cos(\omega t)\exp(-\frac{2r^2}{w(z)^2})
  \label{eq:1d_lattice_efield_exact},
\end{equation}
wherein $r$ denotes the beam radius, $w(z)$ the spot size parameter and $k$
the angular wavenumber of the laser. The wavenumber relates to the laser
wavelength via $k=2\pi/\lambda$. The magnetic field component is chosen
in such a way that it satisfies the Maxwell equations but will not be
considered relevant any further as the electron only has a small magnetic
dipole. Analogue to \cite{Rom2009} we will now assume $w(z)\gg r$ and drop
the exponential term in \cref{eq:1d_lattice_efield_exact}, thus simplifying
\cref{eq:1d_lattice_efield_exact} to
\begin{equation}
  \vb{E}(t)
  \approx\vb{E}_0\cos(kz)\cos(\omega t)
  \label{eq:1d_lattice_efield_approx}.
\end{equation}
Note that even the electric field in \cref{eq:1d_lattice_efield_approx} has
a spatial dependency, the dipole approximation still holds in the atomic
reference frame.

\subsection{Effective dipole potential}

We are now going to solve \cref{eq:hamiltonian_dip} for the case of an
electric field of the form \cref{eq:1d_lattice_efield_approx} with
for-off-resonant laser wavelength $\lambda$ and finally derive an expression
for the effective dipole potential.

At $t<0$ the system is in the energy eigenstate $\ket{n}$ of the unperturbated
Hamiltonian
\begin{equation}
  \hat{H}_0\ket{n}
  =E_n\ket{n}
  =\hbar\omega_n\ket{n}
  \label{eq:eigenvalue_energy_unperturbated}.
\end{equation}
At $t>0$ the external light field appears instantaneous. The new state
$\ket{\psi}$ can be expanded in the complete base of the previous energy
eigenstates
\begin{equation}
  \ket{\psi}
  =\sum_nc_n(t)e^{-i\omega_nt}\ket{n}
  \label{eq:state_expansion_unperturbated}.
\end{equation}
Inserting \cref{eq:state_expansion_unperturbated} into the time-dependent
Schrödinger equation with dipole Hamiltonian \cref{eq:hamiltonian_dip} and
applying $\bra{m}e^{i\omega_mt}$ to the right hand side leads us to a set
of differential equations
\begin{equation}
  \dot{c}_m
  =-\frac{i}{\hbar}\sum_nc_n(t)e^{-i\omega_{nm}t}\bra{m}\hat{V}_\text{dip}\ket{n}
  \label{eq:differential_equation_population_dynamics}
\end{equation}
with $\omega_{nm}=\omega_n-\omega_m$, that describe the dynamics of the
probablity amplitudes $c_n(t)$. By using the electric field we derived in
\cref{eq:1d_lattice_efield_approx} we can rewrite the dipole transition
matrix elements
\begin{equation}
  \bra{m}\hat{V}_\text{dip}\ket{n}
  =\hbar\Omega_{nm}\cos(kz)\cos(\omega t)
  \label{eq:elements_dipole_transition_matrix}
\end{equation}
where we introduced the Rabi frequency
$\Omega_{nm}=\bra{n}\hat{\vb{d}}\cdot\vb{E}_0\ket{m}/\hbar$. Explicit values
for general dipole transition elements for one and two electron systems can
be found in \cite{Bethe1957}. Als the transition elements vanishes for states
with same parity, in particular do we have $\Omega_{nn}=0$
\cite{Bartelmann2018}.

From now on we assume a two state system that initially is only populated in
the ground state $c_g(0)=1,c_e(0)=0$ and the dynamics in
\cref{eq:differential_equation_population_dynamics} are simply described by
\begin{align}
  i\dot{c}_g=\Omega_0c_e(t)\cos(kz)\cos(\omega t)e^{+i\omega_0 t} &&
  i\dot{c}_e=\Omega_0c_g(t)\cos(kz)\cos(\omega t)e^{-i\omega_0 t}
  \label{eq:differential_equation_population_dynamics_two_state_system}.
\end{align}
Expansion of $\cos(\omega t)$ in terms of the exponential function and
dropping $e^{\pm i(\omega+\omega_0)t}$ yields
\begin{align}
  i\dot{c}_g\approx\frac{\Omega_0}{2}c_e(t)\cos(kz)e^{+i\Delta\omega t} &&
  i\dot{c}_e\approx\frac{\Omega_0}{2}c_g(t)\cos(kz)e^{-i\Delta\omega t}
  \label{eq:differential_equation_population_dynamics_two_state_system_rwa}.
\end{align}
The so-called rotating wave approximation is motivated by the fact that
oscillations of frequency $\omega+\omega_0$ are fast compared to changes in
the population dynamics and therefore vanish on average.

We now define $a_g=c_g$ and $a_e=c_ee^{i\Delta\omega t}$ and refine
\cref{eq:differential_equation_population_dynamics_two_state_system_rwa} by
\begin{align}
  i\dot{a}_g=\frac{\Omega_0}{2}a_e(t)\cos(kz) &&
  i\dot{a}_e=\frac{\Omega_0}{2}a_g(t)\cos(kz)-a_e\Delta\omega
  \label{eq:differential_equation_population_dynamics_two_state_system_shift}.
\end{align}
In the former form we can diagonalize the Hamiltonian and find energy
eigenvalues to be
\begin{equation}
  E_{e,g}
  =\frac{\hbar}{2}\left(-\Delta\omega\mp\sqrt{\Omega_0^2+\Delta\omega^2}\right)
  \approx
  =\mp\frac{\hbar\Omega_0^2}{4\Delta\omega}
  \label{eq:eigenvalues_energy_light_shift}
\end{equation}
where we applied Taylor expansion to the square root for
$\Delta\omega\gg\Omega_0$.

In summary atoms in an external off-resonant light field with experience an
effective periodic dipole potential
\begin{equation}
  \hat{V}_\text{eff}
  =V_0\cos^2(kz)
  \qc
  V_0
  =\frac{\hbar\Omega}{4\Delta\omega}\cos^2(kz)
  =\frac{\hbar I_0\cos^2(kz)}{2c_0\epsilon_0\Delta\omega}
\end{equation}
with vacuum speed of light $c_0$, dielectric constant $\epsilon_0$ and
maximum intensity $I_0$. Same results can also be obtained by the use of
second order perturbation theory, however the presented approach is more
clear on assumptions and time dependence \cite{Grimm2008}.

\section{Harmonic approximation}

Another approach to solve \cref{eq:hamiltonian_lattice} in closed form is to
use the fact that cooled atoms have a low kinetic energy and will be located
at the bottom of the potential wells $\hat{V}_\text{lat}$ such that we can
do an harmonic approximation
\begin{equation}
  V_\text{lat}(x)
  =V_0\cos^2(kx)
  \approx V_0\left(1-(kx)^2\right)
  \label{eq:potential_lattice_approximation}
\end{equation}
inserting \cref{eq:potential_lattice_approximation} into
\cref{eq:hamiltonian_lattice} and rearranging yield us the Hamiltonian of
the harmonic oscillator
\begin{equation}
  \hat{H}_\text{har}-V_0
  =\left(\frac{\hat{p}^2}{2m}-V_0k^2x^2\right)\ket{\phi}
  \label{eq:hamiltonian_harmonic}
\end{equation}
with well-known energy levels $E_n=\hbar\omega(2n+1)/2$ and frequency
$\omega=\sqrt{V_0/m}k=\sqrt{2V_0E_r}/\hbar$. In the harmonic approximation
the energy bands found in \cref{eq:eigenvalue_energy_energy_lattice_final}
decrease to single energy levels. By contrast flat potentials would increase
the width of the energy bands which in the limit leads to free atoms. We are
going to use the harmonic approximation to cross check later results.

\section{States in optical lattice}

With the previous foundations in place we can proceed with the solid state
analogues for optical lattices. From the  the simple analytic form of the
optical potential, \cref{eq:potential_lattice}, we are able to
derive an expression for the quantum tunneling probability which gives us
an upper bound for the duration time of our potential generation.

\subsection{Bravais lattice}

A Bravais lattice of dimension $N$ consists of the discrete points
\begin{equation}
  \vb{r}
  =\bra{\vb{x}}\ket{\vb{r}}
  =\sum_{i=1}^Nn_i\vb{a}_i
  \qc n_1,\dots,n_N\in\mathbb{Z}
\end{equation}
where the $\vb{a}_i$ are the primitive vectors that span the primitive cell.
The reciprocal lattice is defined as the point set
\begin{equation}
  \vb{g}
  =\bra{\vb{x}}\ket{\vb{g}}
  =\sum_{i=1}^Nm_i\vb{b}_i
  \qc m_1,\dots,m_N\in\mathbb{Z}
\end{equation}
that satisfy the condition $\exp(\vb{r}\cdot\vb{g})=1$. For a simple cubic
lattice structure $\vb{b}_i=2\pi\vb{a}/{{\vb{a}_i}^2}$ satisfies said
condition.

In the reciprocal lattice space it is especially simple to express periodic
functions. This suggests the transformation of a periodic potential
\cref{eq:potential_lattice} into the reciprocal lattice space.
\begin{equation}
  \bra{g_l}\hat{V}\ket{g_{l^\prime}}
  &=\int\dd{x}\bra{g_l}\ket{x}V(x)\overline{\bra{g_{l^\prime}}\ket{x}}
  \label{eq:transform_reciprocal_lattice}
\end{equation}
With $\bra{x}\ket{g_l}=e^{2ikxl}/\sqrt{2\pi}$ we can identify
\cref{eq:transform_reciprocal_lattice} as Fourier transform into the
reciprocal lattice base. Taking \cref{eq:potential_lattice} as periodic
potential and using orthogonality between Fourier modes we find
\begin{equation}
  \bra{g_l}\hat{V}_\text{lat}\ket{g_{l^\prime}}
  =\frac{1}{4}V_0\left(\delta^l_{l^\prime-1}+\delta^l_{l^\prime+1}+2\delta^l_{l^\prime}\right)
  \label{eq:potential_reciprocal_lattice_base}.
\end{equation}

\subsection{Bloch states}

Bloch's theorem states that for a periodic potential
\begin{equation}
  V_\text{per}(\vb{x}+\vb{a})=V_\text{per}(\vb{x})
  \label{eq:potential_periodic}
\end{equation}
there exists a complete set of wavefunctions that are energy eigenstates
of the Hamiltonian
\begin{equation}
  \hat{H}_\text{per}
  =\frac{\hat{\vb{p}}^2}{2m}+\hat{V}_\text{per}
  \label{eq:hamiltonian_periodic},
\end{equation}
and each of these Bloch waves can be written in the form
\begin{equation}
  \bra{\vb{x}}\ket{\Psi^n_{\vb{q}}}
  =\Psi^n_{\vb{q}}(\vb{x})
  =e^{i\vb{q}\cdot\vb{x}}\psi^n_{\vb{q}}(\vb{x})
  \label{eq:wave_bloch}
\end{equation}
with $\psi^n_{\vb{q}}(\vb{x}+\vb{a})=\psi^n_{\vb{q}}(\vb{x})$, wave vector
$\vb{q}$ and bandindex $n$. We confine the wave number to the first Brillouin
zone $-k<q<+k$ to obtain an unique set of Bloch waves. For a proof of Bloch's
theorem see \cite{Roessler2004} or \cite{Bartelmann2018}.

\subsection{Energy band structure}

Using the Bloch waves \cref{eq:wave_bloch} as ansatz for the Hamiltonian
\begin{equation}
  \hat{H}_\text{lat}
  =\frac{\hat{\vb{p}}}{2m}+\hat{V}_\text{lat}
  \label{eq:hamiltonian_lattice},
\end{equation}
$V_\text{lat}(z)=V_0\cos^2(kz)$ found in \cref{eq:potential_periodic}, we are
now able to find an expression for the energy eigenvalues. We notice that by
the product rule
$\hat{p}^2\ket{\Psi^n_{q}}=e^{iqx}(\hat{p}+\hbar q)\ket{\psi^n_{q}}$,
thus finding
\begin{equation}
  E^n_q\ket{\Psi^n_{q}}
  =\hat{H}_{lat}\ket{\Psi^n_{q}}
  =\left(\frac{(\hat{p}+\hbar q)^2}{2m}+\hat{V}_\text{lat}\right)\ket{\psi^n_{q}}
  \label{eq:eigenvalue_energy_lattice}.
\end{equation}

We expand the state $\ket{\psi^n_{q}}$ in the reciprocal lattice base
\begin{equation}
  \ket{\psi^n_{q}}
  =\left(\sum_l\ket{g_l}\bra{g_l}\right)\ket{\psi^n_{q}}
  =\sum_l\braket{g_l}{\psi^n_{q}}\ket{g_l}
  =\sum_lc_l\ket{g_l}
  \label{eq:state_expansion_reciprocal_lattice_base}.
\end{equation}
We need to find the momentum eigenvalues of the state $\ket{g_l}$
\begin{equation}
  \hat{p}\ket{g_l}
  =\int\int\dd{x}\dd{y}\ket{y}\bra{y}\hat{p}\ket{x}\bra{x}\ket{g_l}
  \label{eq:eigenvalue_momentum_reciprocal_lattice1}.
\end{equation}
With $\bra{y}\hat{p}\ket{x}=-i\hbar\delta(y-x)\dv{x}$ we can take the
derivative of $\bra{x}\ket{g_l}=e^{2ikxl}$ and simplify
\cref{eq:eigenvalue_momentum_reciprocal_lattice1} to
\begin{equation}
  \hat{p}\ket{g_l}
  =2\hbar kl\int\dd{x}\ket{x}\bra{x}\ket{g_l}
  =2\hbar kl\ket{g_l}
  \label{eq:eigenvalue_momentum_reciprocal_lattice1}.
\end{equation}

Finally we insert \cref{eq:state_expansion_reciprocal_lattice_base} into
\cref{eq:eigenvalue_energy_lattice} and apply $\bra{g_{l^\prime}}$ to the
right hand side while using \cref{eq:potential_reciprocal_lattice_base} and
\cref{eq:eigenvalue_momentum_reciprocal_lattice1}, yielding
\begin{equation}
  E^n_q
  =\bra{g_{l}}\hat{H}_\text{lat}\ket{g_{l^\prime}}
  =
  \begin{cases}
    (l+q/k)^2E_r+\frac{1}{2}V_0, & \text{if } l=l^\prime\\
    \frac{1}{4}V_0, & \text{if } \abs{l-l^\prime}=1\\
    0, & \text{otherwise}
  \end{cases}
  \label{eq:eigenvalue_energy_energy_lattice_final}
\end{equation}
wherein we introduced the recoil energy $E_r=\hbar^2k^2/(2m)$ which gives us
a natural, atom independent energy scale.

\subsection{Wannier states}

So far we studied the effects of a periodic potential on the energy levels of
a wave function that propagates over the complete lattice. Lattice hopping
however is a local effect, therefore we need a spatially localized set of
wave functions to describe the effect of lattice hopping. Fortunately the
so called Wannier functions provide such an localized function base.

The orthodox definition of the Wannier function with bandindex $n$ at lattice
site $l$ reads
\begin{equation}
  \bra{x}\ket{\phi^n_{l}}
  =\phi^n_{l}(x)
  =\frac{1}{\sqrt{N}}\sum_qe^{-iqla}\Psi^n_{q}(x)
  \label{eq:wave_wannier_orthodox}
\end{equation}
wherein $\Psi^n_{q}$ is the Bloch wave from \cref{eq:wave_bloch}, $N$ the
total number of lattice sites and the sum over $q$ is confined to the first
Brillouin zone with $q$ that satisfy the periodic boundary condition $q=2ks/N$
with integer $s\in\mathbb{Z}$.

\section{Hopping frequency}

Quantum mechanics predicts a nonzero probability for particles to tunnel
through an energy barrier. Lattice hopping describes this phenomena of atoms
tunneling through the optical lattice potentials to other lattice sites. In
this section we want to find an expression for the probability amplitude of
such hopping effects.


