\begin{abstract}
Ultracold atoms in optical lattices open up a wide range of possibilities
to simulate many-body quantum phenomena, which would elsewise be neither
computationally nor experimentally tangible.

The topology of the optical lattice is a decisive property of these kind
of experiments and therefore of major interest. Recently, efforts have been
reported to create novel optical potentials through the use of digital
micromirror arrays that permit alterations of localized potentials. Though
promising results have been achieved --- in particular for static potentials
--- limitations due to the mechanical nature of these mirror arrays arise,
for instance, with regard to dynamical control.

In the present work we present an alternative implementation of localized
optical lattice potentials based on acousto-optic deflectors and
direct digital synthesizers.

We will give a brief theoretical introduction into the main concepts, i.e.\
the interaction of neutral atoms with optical potentials as well as the
general operation of direct digital synthesizer and acousto-optic deflectors.
From the physics we can derive the requirements imposed on the technical
implementation of the \gls{rf} signal source and the deflection.

We will find that even though the platform of digital signal synthesis
generally suites our application in terms of modulation capabilities and
resolution, the particular implementation of the \gls{ad9910} demonstrates
several shortcomings.

In the second part we characterize the deflection efficiency of the
acousto-optic deflectors. Towards the end we try to minimize the variance of
the deflection efficiency by performing a random search on the amplitude
segments of the \gls{rf} signal. It turns out that the deflection efficiency
is a highly non-linear function of the applied \gls{rf} power and frequency.
Furthermore minimization of the deflection efficiency variances proves to be
very unstable. Though we can largely preclude electronic defects to be the
source of this behaviour, further investigation is required.
\end{abstract}
