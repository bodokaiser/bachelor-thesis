\begin{abstract}
  Ultracold atoms in optical lattices open up a wide range of possibilities
  to simulate quantum effects which would elsewise be neither computational
  nor experimentally tangible.

  The topology of the optical lattice is a decisive parameter of these kind
  of experiments and therefore of major interest. Recently, efforts have been
  reported to support novel potential dynamics through the use of micromirror
  arrays that permit alterations of localized potentials. Though promising
  results have been achieved, limitations due to the mechanical nature of
  these mirror arrays are well-known.

  In the present work we present an alternative low-cost implementation of
  localized optical lattice potentials based on acousto-optic deflectors and
  direct digital synthesizers.

  We will discuss theoretical aspects like the interaction of neutral atoms
  with an optical potential as well as the general operation of a direct
  digital synthesizer and an acousto-optic deflector. In the experimental
  chapters we will characterize the radio frequency signal propagation from
  the synthesizer to the acousto-optics plus the intensity transmission
  of the acousto-optic deflectors.

  We will find that though this approach proves to be feasible in general,
  the practical application places requirements on the electronic signal
  source not met by the direct digital synthesizers. The encountered
  challenges and conclusions drawn however, are of considerable importance
  for a foreseen design of an apparatus to create localized optical lattice
  potentials.
\end{abstract}
