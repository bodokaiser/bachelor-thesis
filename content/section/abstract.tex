\begin{abstract}
Ultracold atoms in optical lattices open up a wide range of possibilities
to simulate many-body quantum phenomena, which would elsewise be neither
computationally nor experimentally tangible.

The topology of the optical lattice is a decisive property of these kind
of experiments and therefore of major interest. Recently, efforts have been
reported to create novel optical potentials through the use of digital
micromirror arrays that permit alterations of localized potentials. Though
promising results have been achieved --- in particular for static potentials
--- limitations due to the mechanical nature of these mirror arrays arise,
for instance, with regard to dynamical control.

In the present work we present an alternative implementation of localized
optical lattice potentials based on acousto-optic deflectors and
direct digital synthesizers.

We will give a brief theoretical introduction into the main concepts, i.e.\
the interaction of neutral atoms with optical potentials as well as the
general operation of direct digital synthesizer and acousto-optic deflectors.
In the second part we will characterize the radio frequency signal propagation from
the synthesizer to the acousto-optic deflector plus the deflection efficiency
of the acousto-optic deflectors.

We will find that though this approach proves to be feasible in general,
the practical application demands stringent requirements on the electronic
signal source not met by the direct digital synthesizers. The encountered
challenges and conclusions drawn, however, are of considerable importance
for a foreseen design of an apparatus to create dynamically-controllable local
optical potentials.
\end{abstract}
