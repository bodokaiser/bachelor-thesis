\chapter{Calibration}

Alterations of the laboratory environment combined with the exchange of
components from the original setup made it advisable to recalibrate the setup.
In this chapter we want to document the steps necessary for calibration for
successive experiments.

\section{Fiber Coupling}

The visually shielded section of the setup used to reduce the output power
of the laser source is optically paired with the open section for beam
deflection via a \gls{smf} that only permits two orthogonal polarization and
a single gaussian mode. Through tuning the polarisator inside the power
reduction section we can try to match one of the orthogonal polarization
modes supported by the \gls{smf}. Polarization discrepancies induce the
polarization inside the \gls{smf} to oscillate on environmental changes like
temperature or vibration. Henceforth it is key to couple polarization modes
in order to ensure a stable operation.

The following receipt has proven to be successful to find an approximate
polarization match between the laser beam and the \gls{smf}. In addition
to the setup described in \ref{sec:powerbox} and \ref{sec:deflection}
an oscilloscope and a hot air gun are highly recommended.

\begin{enumerate}
  \item Connect the photodiode to the oscilloscope and use a coarse time
    scale of i.e. \SI{2}{\second}.
  \item Apply safety measures i.e. put on appropiete laser safety glasses and
    inform present personal of the imminent danger.
  \item Open the cover of the power reduction setup.
  \item Apply heat to the \gls{smf} through the hot air gun, alternatively
    you can try to move the fiber.
  \item The photodiode signal should start to oscillate. Tune the polarizor
    inside the power reduction subject to minimizing the oscillation.
\end{enumerate}

The oscillations occur as the polarization circulates inside the fiber and
will stop at some point. In this case yu should remove the heat or mechanical 
stress on the fiber and wait some time before you can reapply with the initial
response.

\section{Beam Alignment}

As is well known, beams that pass off-centered through spherical lenses
experience optical aberrations. Additionally we found that uncentered beams
may cause further optical defects from reflections at boundaries. Both effects
should be avoided as far as possible by adjusting positions and mirror angles.

As auxilliaries we recommend a pair of iris diaphragms that are mountable to
the lens mounts and a screen i.e. a white sheet of hard paper. With the iris
diaphragms we can easily find a center reference point visually by inspecting
the symmetry of the iris illumination at different pinhole diameters. Exact
steps are summarized in the below receipt. We refer to the optics as annotated
in Figure~\ref{fig:deflection}.

\begin{enumerate}
  \item Place screen in front of L2 and adjust screws on fiber coupler subject
    to a clean beam profile on the screen of the (1,1) diffraction order of
    the \gls{aom}s.
  \item Apply one iris to L2 and find center frequency of the \gls{aom}s.
  \item Mount irises on L3 and L4. Mount mirror pair M1, M2 to direct beam
    through both pinholes.
  \item Place auxilliary lens with mounted iris between L7 and M3 and adjust
    second objective pair L6, L7.
  \item Mount iris on L8 and place auxilliary lens with mounted iris in front
    of BS3. Align the mirror pair M3 and M4 to reflect the beam through both
    irises.
  \item Adjust the objective pair distance until you see an output similar to
    XY on the \gls{ccd} camera.
\end{enumerate}

The alignment of a mirror pair can be simplified by using the below algorithm:

\begin{enumerate}
  \item Select axis to align.
  \item Align the beam on the outside lens by tuning the inner mirror.
  \item Align the beam on the inner lens by tuning the outside mirror.
  \item Repeat steps 2 and 3 until beam is aligned on both lenses. Proceed
    with other axis.
\end{enumerate}

\section{Camera Focus}

Finally we had to reposition the camera to focus the incoming beam correctly
in order to register the beam profile as would be seen later by the atoms.
To find the precise focus position we followed the procedure described in
\cite{Hertlein2017} that consits of extracting the camera rail with its lens
and focusing it on a far distant object.

\begin{figure}[ht]
  \centering
  \includegraphics[width=\textwidth]{\builddir{focus.pdf}}
  \caption{Camera focused on far distant object. In this case a window from the
  building accross the courtyard.}
\end{figure}
