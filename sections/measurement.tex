\chapter{Measurement}

With the previously described experimental setup we conducted various
measurements. We can classifier these measurements into two sets: The first
class includes measurements for calibration, the second measurements with the
intent to prove concepts and benchmark the results.

\section{Calibration}

Alterations inside the laboratory together with the exchange of components
from the original setup made it necessary to recalibrate the optics.

\subsection{Fiber Coupling}

The visually shielded section of the setup used to reduce the output power
of the laser source is optically paired with the open section for beam
deflection via a \gls{smf}. Through tuning the polarisator inside the power
reduction section we can try to match one of the orthogonal polarization
modes supported by the \gls{smf}. A polarization discrepancy induces the
polarization inside the \gls{smf} to oscillate on environmental changes like
temperature, thus we tried to minimize these effects by observing intensity
changes detected by a photodiode behind a beam splitter through an
oscilloscope.

\subsection{Beam Alignment}

As is well known, beams that pass off-centered through spherical lenses
experience optical aberrations. Additionally we found that uncentered beams
may cause further optical defects from reflections at boundaries. Both effects
should be avoided as far as possible by adjusting positions and mirror angles.
As auxilliaries we used a pair of iris diaphragms that are mountable to
most optics we used as well as a post with a crosshair to find the correct
center point.

\subsection{Camera Focus}

Finally we had to reposition the camera to focus the incoming beam correctly
in order to register the beam profile as would be seen later by the atoms.
To find the precise focus position we followed the procedure described in
\cite{Hertlein2017} that consits of extracting the camera rail with its lens
and focusing it on a far distant object.

\begin{figure}[h]
  \centering
  \includegraphics[width=\textwidth]{\builddir{focus.pdf}}
  \caption{Camera focused on far distant object. In this case a window from the
  building accross the courtyard.}
\end{figure}
