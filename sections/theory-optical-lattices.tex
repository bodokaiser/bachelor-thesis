\chapter{Theory of Optical Lattices}

In this chapter we will recapitulate the theoretical foundations of optical
lattices used in ultracold quantum experiments, thereof we can derive
theoretical requirements for the planned potential generation. Most notably
we are interested in the domain of the sweep duration and resolution that
our setup has to operate in.

\section{Periodic Potentials}

The Schrödinger equation of a particle state $\ket{\Psi}$ subject to a time
independent one-dimensional potential reads
\begin{align}
  E\ket{\Psi}
  =\hat{H}\ket{\Psi}
  =\left(\frac{\hat{p}^2}{2m}+\hat{V}\right)\ket{\Psi}
  \label{eq:time_independent_schroedinger}.
\end{align}
For a periodic potential $V(x+a)=V(x)$ the Bloch-Theorem \cite{Ashcroft1976}
states that the wavefunction of a particle in a periodic potential can be
written as
\begin{equation}
  \braket{\hat{x}}{\Psi}
  =\Psi(x)
  =e^{iqx}\psi(x)
  \label{eq:bloch_theorem}
\end{equation}
wherein $-\frac{\pi}{a}<q<+\frac{\pi}{a}$ is the quasimomentum, a quantum
number characteristic of the translational symmetry of the periodic potential
\cite[p. 42]{Lewenstein2012} with $a$ being the lattice constant.

The momentum operator is a differential operator in position space
\begin{equation}
  \hat{p}
  =\expval{\hat{p}}{\hat{x}}
  =-i\hbar\dv{x}
  \label{eq:momentum_operator_in_position_space},
\end{equation}
thus we obtain $\hat{p}e^{iqx}=\hbar qe^{iqx}$ and
$\hat{p}^2\ket{\Psi}=e^{iqx}\left(\hat{p}+\hbar q\right)^2\ket{\psi}$ such
that we can rewrite \cref{eq:time_independent_schroedinger} for
the periodic case
\begin{equation}
  E\ket{\psi}
  =\left(\frac{(\hat{p}+\hbar q)^2}{2m}+\hat{V}\right)\ket{\psi}
  =:\hat{H}_{periodic}\ket{\psi}
  \label{eq:periodic_schroedinger}.
\end{equation}

Translational symmetry of the potential enforces discrete particle momenta
$p_n=pn=nk\hbar$ such that we can represent our particle state in a
countable momentum basis
\begin{equation}
  \ket{\psi}
  =\mathbb{1}\ket{\psi}
  =\left(\sum_n\ket{p_n}\bra{p_n}\right)\ket{\psi}
  =\sum_n\braket{p_n}{\psi}\ket{p_n}
  =\sum_nc_n\ket{p_n}
  \label{eq:countable_momentum_state}.
\end{equation}
Inserting this into the right hand side of \cref{eq:periodic_schroedinger}
yields us
\begin{equation}
  \hat{H}_\text{periodic}\ket{\psi}
  =\sum_nc_n\left(\frac{(\hat{p}+\hbar q)^2}{2m}+\hat{V}\right)\ket{p_n}
  =\sum_nc_n\left(\frac{(pn+\hbar q)^2}{2m}+\hat{V}\right)\ket{p_n}
  \label{eq:periodic_hamiltonian_in_momentum_base}
\end{equation}
where we used the eigenvalue equation $\hat{p}\ket{p_n}=pn\ket{p_n}$ to
satisfy the second equal sign. By applying an orthogonal $\bra{p_m}$ to the
left hand side of \cref{eq:periodic_hamiltonian_in_momentum_base} we find
the matrix elements of the periodic Hamiltonian
\begin{equation}
  \hat{H}_{\text{periodic},mn}
  =\mel**{p_m}{\hat{H}_\text{periodic}}{p_n}
  =(m+q/k)^2\frac{\hbar^2k^2}{2m}+\mel**{p_m}{\hat{V}}{p_n}
  \label{eq:periodic_hamiltonian_momentum_elements}
\end{equation}
where we used $p=\hbar k$ in the last step.

Usually the potential will be known in position space. Through use of the
completeness relation
\begin{equation}
  \mathbb{1}
  =\int\dd{x}\op{x}{x}
  \label{eq:continous_completeness_relation}
\end{equation}
we are able to transform the potential in momentum space
\begin{equation}
  V(p)
  =\mel**{p_m}{\hat{V}}{p_n}
  =\int\int\dd{x}\dd{y}\braket{p_m}{x}\mel**{x}{\hat{V}}{y}\braket{y}{p_n}
  =\int\frac{\dd{x}}{\sqrt{2\pi\hbar}}e^{ikx(m-n)}V(x)
  \label{eq:potential_fourier_transform}.
\end{equation}
after the identification of $\braket{p}{x}=e^{ipx/\hbar}/\sqrt{2\pi\hbar}$.

\section{Optical Potentials}

By perpendicular reflection of a single Gaussian laser beam, one can construct
a standing wave yielding a one dimensional optical lattice with potential
\begin{equation}
  V(r,x)
  =V_0e^{-2r^2/w(x)^2}\cos^2(kx)
  \approx V_0\cos^2(kx)
  \label{eq:optical_potential}
\end{equation}
with $r$ being the beam radius, $w(x)$ being spot size parameter and
$k=2\pi/\lambda$ with the laser wavelength $\lambda$. The approximation
is valid for $w(x)\gg r^2$ which can be assumed true in most cases
\cite[p.127]{Rom2009}. The lattice constant of such a potential is given
by $a=\lambda/2$.

Given the momentum representation of \cref{eq:optical_potential} can be found
with \cref{eq:potential_fourier_transform} to be
\begin{align*}
  V(p)
  &=\int\frac{\dd{x}}{\sqrt{2\pi\hbar}}e^{ikx(m-n)}V(x)\\
  &=
  \frac{1}{4}V_0
  \int\frac{\dd{x}}{\sqrt{2\pi\hbar}}
  e^{+ipx(m-n)/\hbar}\left(e^{+2ikx}+e^{-2ikx}+2\right)\\
  &=
  \frac{1}{4}V_0\left(2\delta_{m,n}+\delta_{m,n+2}+\delta_{m,n-2}\right)
\end{align*}
and the Hamiltonian \cref{eq:periodic_hamiltonian_momentum_elements} can be
specified as
\begin{equation}
  \hat{H}_\text{optical}
  =\begin{cases}
    (n+q/k)^2\frac{\hbar^2k^2}{2m}+\frac{1}{2}V_0, & \text{if } n=m\\
    \frac{1}{4}V_0, & \text{if } \abs{n-m}=2\\
    0, & \text{otherwise}
  \end{cases}
  \label{eq:optical_hamiltonian}
\end{equation}
where we can identify the recoil energy $E_r=(\hbar k)^2/(2m)$. The recoil
energy is used to gather an energy scale independent from the specific atomic
properties. Plotting the eigenvalues of \cref{eq:optical_hamiltonian} against
$q$ we find the characteristic energy band structure giving us the allowed
energy ranges.

\subsection{Tight-Binding Approximation}

In the tight-binding approximation the potential $V_0$.
