\chapter{Experimental Setup}

The exerimental setup was largely adopted from \cite{Hertlein2017}. Amendments
made include the \gls{rf} signal source of the \gls{aod} and an additional
photodiode to measure the deflected beam intensity.

\section{Optics}

The optical setup can be dissect into two sections. The first section reduces
the power of the $\SI{532}{\nano\meter}$ laser source from $\SI{10}{\watt}$
to below $\SI{2}{\milli\watt}$ for the purpose of safe operation of an open
deflection setup, the second section. Both sections are connected through a
\gls{smf}.

\subsection{Power Reduction}
\label{sec:powerbox}

For safety reasons the power reduction setup is confined into a visually
sealed superstructure. Figure~\ref{fig:powerbox} reveals the inside of the
power reduction box.

\begin{figure}[h]
  \centering
  \includegraphics[width=\textwidth]{\builddir{powerbox.pdf}}
  \caption{Optical configuration of the power reduction box.}
  \label{fig:powerbox}
\end{figure}

The laser beam leaving the laser source is polarized by a $\lambda/2$ retarder
plate such that the succeeding high power beamsplitter BS can divert a large
part of the power into a beam dump.

Afterwards mirrors M1 and M2 direct the beam towards the center of a
$2:1$ telescope composed of two lenses L1, L2 with focal length
$f_1=\SI{100}{\milli\meter}$ and $f_2=\SI{50}{\milli\meter}$.

An \gls{aom} diffracts the laser collimated laser beam into multiple orders.
Mirrors M3, M4 project theses orders onto a pinhole which is adjusted to
absorb all orders except for the first order. The intensity of the first order
is subject to the amplitude signal apllied to the \gls{aom}.

Finally a tunable $\lambda/2$ retarder plate can be used to couple the beam
polarization with the \gls{smf}.

\subsection{Beam Deflection and Detection}
\label{sec:deflection}

The setup for beam deflection and detection receives the down-powered laser
beam of the previously described section by a \gls{smf}.
Figure~\ref{fig:deflection} discloses the arrangement used in said setup.
Hereinafter the beam leaving the \gls{smf} passes a tunable retarder plate
and beam splitter BS1. The tunable retarder plate can be used to adjust the
beam intensity without having to access the power box.
A second polarizer with cube BS2 is used to branch off a part of the beam
to a photodiode PD1 that is positioned to be at the focal point of lens L1.

\begin{figure}[h]
  \centering
  \includegraphics[width=\textwidth]{\builddir{deflection.pdf}}
  \caption{Optical configuration of the main setup section.}
  \label{fig:deflection}
\end{figure}

Two \gls{aom} are used for vertical and horizontal beam deflection as a
function of applied frequencies. From hereon a $1:1$ telescope comprised
of two lenses L2, L3 with focal length $f_2=f_3=\SI{250}{\milli\meter}$
projects the laser beam on a pair of objectives that are built of lenses
L4 to L7. The objectives will be used later to focus the beam on to the atom
lattice.

Further on the laser is reflected by a pair of mirrors M3 and M4 to the part
intended for detection. Lens L8 acts as a camera lens and projects the beam
to infinite focus. Cube BS3 forks a portion of the beam away from the CCD
camera on to mirror M5. Mirror M5 reflects the beam towards lens L9 that
focuses the beam onto a second photodiode PD2.

\section{Electronics}

Beforehand we described the optical setups used. Now we want to emphasize
on the conjunction of hybrid components, i.e. photodiods and \gls{aom}s, with
the present electronic setup.

In the following we will refer to \gls{aom} as the \gls{aom} in the power
reduction optical setup with the purpose of intensity control against laser
source and temperature fluctuations. With \gls{aom} H and \gls{aom}
V we will refer to the \gls{aom}s used for beam deflection in the
second optical setup.

Listed \gls{aom} are driven by a \gls{rf} signal. In the case of \gls{aom} I
the amplitude modulation corresponds to intensity modulation of the deflected
orders at constant frequency $f_I=\SI{80}{\mega\hertz}$. \gls{aom} H
and \gls{aom} V deflect the first order diffraction proportional to
a frequency signal deviation around $f_H=f_V=\SI{100}{\mega\hertz}$. Further
we can compensate for intensity losses by deflection through additional
amplitude modulation as described for \gls{aom} I.

\subsection{Signal Source}

The requirements placed by the \gls{aom} demand a flexible but precise
\gls{rf} source. Thankfully we can resort to a custom made signal source
based on the \cite{AD9910} direct-digital synthesizer that can operate up to
\SI{420}{\mega\hertz} and allow modulation of amplitude, frequency and phase
offset by either a constant value or through a single digital ramp or playback
from memory.

\subsection{Intensity Control}

\subsection{Deflection Control}

\subsection{Trigger Control}
