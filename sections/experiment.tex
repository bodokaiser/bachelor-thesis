\chapter{Experimental Setup}

The exerimental setup was largely adopted from \cite{Hertlein2017}, amendments
made include the \gls{rf} signal source of the \gls{aod} and an additional
photodiode to measure the deflected beam intensity.

\section{Optics}

The optical setup can be dissect into a closed first section that reduces
the power of the $\SI{532}{\nano\meter}$ laser source from $\SI{10}{\watt}$
to below $\SI{2}{\milli\watt}$ and an open second section for beam deflection.
Both sections are connected through a \gls{smf}.

\subsection{Power Reduction}
\label{sec:powerbox}

Because of safety concerns the power reduction section is confined into a
visually sealed superstructure. \Cref{fig:powerbox} reveals the inside of the
power reduction box.

\begin{figure}[h]
  \centering
  \includegraphics[width=\textwidth]{build/powerbox.pdf}
  \caption{Optical configuration of the power reduction section.}
  \label{fig:powerbox}
\end{figure}

The laser beam leaving the laser source is polarized by a $\lambda/2$ retarder
plate such that the succeeding high power beamsplitter BS can divert the
majority of the beams power into a high power beam dump.
Afterwards mirrors M1 and M2 direct the beam towards the center of a $2:1$
telescope composed of two lenses L1, L2 with focal lengths
$f_1=\SI{100}{\milli\meter}$ and $f_2=\SI{50}{\milli\meter}$.
An \gls{aom} diffracts the laser beam into multiple orders. Mirrors M3, M4
project theses orders onto a pinhole which is configured to intromit only the
the first order deflection. The intensity of the first order is subject to
amplitude modulation apllied to the \gls{aom}.
Finally a tunable $\lambda/2$ retarder plate can be used to couple the beam
polarization with the \gls{smf}.

\subsection{Beam Deflection and Detection}
\label{sec:deflection}

The section for beam deflection and detection as disclosed in
\cref{fig:deflection} receives the down-powered laser beam from previously
described section by a \gls{smf}. Hereinafter the beam passes a tunable
retarder plate and beam splitter BS1. The tunable retarder plate can be used
to adjust the beam intensity without having to access the power box.
A second polarizer with cube BS2 is used to branch off a part of the beam
to a photodiode PD1 that is positioned to be at the focal point of lens L1.
Photodiode PD1 is connected via a control system with the amplitude
modulation of the \gls{aom} depicted in \cref{fig:powerbox} to stabilize
the laser intensity against i.e. thermal drifts.

\begin{figure}[h]
  \centering
  \includegraphics[width=\textwidth]{build/deflection.pdf}
  \caption{Optical configuration of the beam deflection section.}
  \label{fig:deflection}
\end{figure}

For horizontal and vertical beam deflection two \gls{aod}s are used. A $1:1$
telescope comprised of two lenses L2, L3
each with focal length $f_2=f_3=\SI{250}{\milli\meter}$ projects the beam
on a pair of objectives that are built of lenses L4 to L7. The purpose of the
objective pair is to focus the laser on to the atom plane.
Consecutively the laser is reflected by a pair of mirrors M3 and M4 to the
part intended for detection. Lens L8 acts as a camera lens and projects the
beam to infinite focus on to the CCD camera sensor. Cube BS3 forks a portion
of the beam away from the CCD camera on to mirror M5 that guides the beam
towards lens L9 in order to focus the beam onto a second photodiode PD2.

\section{Electronics}

Beforehand we described the optical setups used. Now we want to emphasize
on the electronics how they are integrated into the optical setup.

\subsection{Signal Source}

The requirements placed by the two \gls{aod} demand two distinct flexible but
precise \gls{rf} source. Thankfully we can resort to a custom made signal
sources based on the \gls{dds} \cite{AD9910} \gls{ic} that can operate up to
\SI{420}{\mega\hertz} and allows modulation of amplitude, frequency and
phase offset by either a constant value or through a single digital ramp or
playback from a memory sequence. A detailed operation of a \gls{dds} can be
found in the \cref{app:elec:dds}.

We apply a reference signal of $f_\text{ref}=\SI{10}{\mega\hertz}$ to the
\gls{dds} which is configured to use a \gls{pll} multiplier of $N=100$
yielding a system clock of $f_\text{sys}=Nf_\text{ref}=\SI{1}{\giga\hertz}$
and a timer clock of $f_\text{timer}=f_\text{sys}/4=\SI{250}{\mega\hertz}$
that is used for the linear ramp and for playback of a memory sequence.

Further the AD9910 \gls{dds} uses \SI{14}{\bit}, \SI{16}{\bit} and
\SI{32}{\bit} internally to parameterize amplitude $A$, phase $\phi$ and
signal output frequency $f_\text{out}$. The playback memory consists of
1024 discrete values for either $A,\phi,f_\text{out}$.

We apply the linear ramp to the frequency in order to deflect the laser beam
in the vertical and horizontal axis. We choose
$f_\text{lower}=\SI{90}{\mega\hertz}$ as lower and
$f_\text{upper}=\SI{110}{\mega\hertz}$ as upper limits for the linear ramp.
The duration of a single sweep is now determined by the ramp rate
$\frac{\Delta f}{\Delta x}$ and the step size $\Delta x$ wherein $\Delta x$
is the counter used internally.


\subsection{Signal Amplifier}


\subsection{Intensity Control}

% short explaination of control loop
% too slow to compensate for intensity deviations in sweep

\subsection{Trigger Source}

To syncronize the signal sources, the \gls{ccd} camera and the oscilloscope
it was necessary to design a global trigger source that outputs a rising edge
signal to multiple devices and exposes a network programable interface.

\begin{figure}[h]
  \centering
  \includegraphics[height=6cm]{example-image-a}
  \caption{The trigger source exposes a network programable interface and
  provides enough power to drive four output signals.}
  \label{fig:elec:trig}
\end{figure}

The schematics and board layout can be found in the \cref{app:elec:trig}.

\section{Acousto-Optics}

\subsection{Modulator}

\subsection{Deflector}
