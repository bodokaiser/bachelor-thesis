\chapter{Experimental Setup}

The exerimental setup was largely adopted from \cite{Hertlein2017}. Amendments
made include the \gls{rf} signal source of the \gls{aod} and an additional
photodiode to measure the deflected beam intensity.

\section{Optics}

The optical setup can be dissect into two sections. The first section reduces
the power of the $\SI{532}{\nano\meter}$ laser source from $\SI{10}{\watt}$
to below $\SI{2}{\milli\watt}$ for the purpose of safe operation of an open
deflection setup, the second section. Both sections are connected through a
\gls{smf}.

\subsection{Power Reduction}

For safety reasons the power reduction setup, Figure~\ref{fig:powerbox},
is confined into a visually sealed superstructure.

\begin{figure}[h]
  \centering
  \includegraphics[width=\textwidth]{\builddir{powerbox.pdf}}
  \caption{Optical configuration of the power reduction box.}
  \label{fig:powerbox}
\end{figure}

The laser beam leaving the laser source is polarized by a $\lambda/2$ retarder
plate such that the succeeding high power beamsplitter BS can divert a large
part of the power into a beam dump.

Afterwards mirrors M1 and M2 direct the beam towards the center of a
$2:1$ telescope composed of two lenses L1, L2 with focal length
$f_1=\SI{100}{\milli\meter}$ and $f_2=\SI{50}{\milli\meter}$.

An \gls{aom} diffracts the laser collimated laser beam into multiple orders.
Mirrors M3, M4 project theses orders onto a pinhole which is adjusted to
absorb all orders except for the first order. The intensity of the first order
is subject to the amplitude signal apllied to the \gls{aom}.

Finally a tunable $\lambda/2$ retarder plate can be used to couple the beam
polarization with the \gls{smf}.

\subsection{Beam Deflection and Detection}

From the \gls{smf} the setup for beam deflection and detection,
Figure~\ref{fig:deflection}, now receives the downpowered beam through a
fiber coupler. Hereinafter a tunable retarder plate with beam splitter BS1
can be used to adjust the beam intensity without having to access the power
box. A second polarizer with cube BS2 is used to branch off part of the beam
to a photodiode PD1. We will describe later how said photodiode is used for
intensity control.

\begin{figure}[h]
  \centering
  \includegraphics[width=\textwidth]{\builddir{deflection.pdf}}
  \caption{Optical configuration of the main setup section.}
  \label{fig:deflection}
\end{figure}

Two \gls{aom} are used for vertical and horizontal beam deflection as a
function of applied frequency pairs. From hereon a $1:1$ telescope comprised
of two lenses L1, L2 with focal length $f_1=f_2=\SI{250}{\milli\meter}$
projects the laser beam on a pair of objectives L3-L6. The objectives will be
used later to focus the beam on to the atom lattice.

The laser beam is reflected by a pair of mirrors M3, M4 to the part intended
for detection. Lens L7 acts as a camera lens and projects the beam to infinite
focus. Cube BS3 forks a portion of the beam away from the CCD camera on to
mirror M5 that reflects the beam towards lens L8 that focuses the beam onto
photodiode PD2.

\section{Electronics}

\subsection{Signal Source}

\subsection{Intensity Control}

\subsection{Deflection Control}
