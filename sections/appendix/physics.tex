\chapter{Physics}

\section{Optical Lattice}

The Schrödinger equation of a particle state $\ket{\Psi}$ subject to a time
independent one-dimensional potential reads
\begin{align}
  E\ket{\Psi}
  =\hat{H}\ket{\Psi}
  =\left(\frac{\hat{p}^2}{2m}+\hat{V}\right)\ket{\Psi}
  \label{eq:physics:schroedinger}.
\end{align}

For a periodic potential $V(x+a)=V(x)$ the Bloch-Theorem \cite{Ashcroft1976}
states that the wavefunction of a particle in a periodic potential can be
written as
\begin{equation}
  \braket{\hat{x}}{\Psi}
  =\Psi(x)
  =\exp(iqx)\psi(x)
  \label{eq:physics:blochtheorem}
\end{equation}
wherein $-\frac{\pi}{a}<q<+\frac{\pi}{a}$ is the quasimomentum, a quantum
number characteristic of the translational symmetry of the periodic potential
\cite[p. 42]{Lewenstein2012} with $a$ being the lattice constant. In the case
of optical lattices with potential
\begin{equation}
  \expval{\hat{x}}{\hat{V}}
  =V(x)
  =V_0\cos^2(kx)
  \label{eq:physics:pospotential}
\end{equation}
with $k=2\pi/\lambda$ the lattice constant is $a=\lambda/2$.

We apply the momentum operator to $\exp(iqx)$ in position space
\begin{equation*}
  \hat{p}\exp(iqx)
  =\expval{\hat{p}}{\hat{x}}\exp(iqx)
  =-i\hbar\dv{x}\exp(iqx)
  =\hbar q\exp(iqx)
\end{equation*}
and use this auxilliary result in order to expand
\begin{equation*}
  \hat{p}^2\ket{\Psi}
  =
  \hat{p}\left(\exp(iqx)(\hat{p}+hq)\right)\ket{\psi}
  =
  \exp(iqx)\left(\hat{p}+\hbar q\right)^2\ket{\psi}
\end{equation*}
inserting the former result and \cref{eq:physics:blochtheorem} into
\cref{eq:physics:schroedinger} we obtain
\begin{equation}
  E\ket{\psi}
  =\left(\frac{(\hat{p}+\hbar q)^2}{2m}+\hat{V}\right)\ket{\psi}
  =\hat{H}^\prime\ket{\psi}
  \label{eq:physics1}.
\end{equation}

Translational symmetry of the potential operator $\hat{V}$ enforces discrete
momenta $p_n=pn=nk\hbar$. We will use this to represent our state in momentum
space through
\begin{equation}
  \ket{\psi}
  =\mathbb{1}\ket{\psi}
  =\left(\sum_n\ket{p_n}\bra{p_n}\right)\ket{\psi}
  =\sum_n\braket{p_n}{\psi}\ket{p_n}
  =\sum_nc_n\ket{p_n}
  \label{eq:physics:psimom}.
\end{equation}
Inserting this into the right hand side of \cref{eq:physics1} yields us
\begin{equation}
  \hat{H}^\prime\ket{p_n}
  =\sum_nc_n\left(\frac{(\hat{p}+\hbar q)^2}{2m}+\hat{V}\right)\ket{p_n}
  \label{eq:physics2}.
\end{equation}
which is satisfied when each term of the sum vanishes. The momentum operator
in momentum space is just the particles momentum, furthermore the
momentum basis is orthogonal. Yielding
\begin{equation}
  \bra{p_n}\frac{(\hat{p}+\hbar q)^2}{2m}\ket{p_m}
  =
  \frac{(pn+\hbar q)^2}{2m}
  =
  \frac{(n\hbar k+\hbar q)^2}{2m}
  =
  (n+q/k)^2\frac{\hbar^2k^2}{2m}
  .
\end{equation}
We still need to find the potential in momentum space
\begin{align*}
  \mel**{p_m}{\hat{V}}{p_n}
  &=
  \int\int\dd{x}\dd{y}\braket{p_m}{x}\mel**{x}{\hat{V}}{y}\braket{y}{p_n}\\
  &=
  \int\int\frac{\dd{x}}{\sqrt{2\pi\hbar}}\frac{\dd{y}}{\sqrt{2\pi\hbar}}
  e^{+ipxm/\hbar}V(y)\delta(y-x)e^{-ipxn/\hbar}\\
  &=
  \int\frac{\dd{x}}{\sqrt{2\pi\hbar}}
  e^{+ipx(m-n)/\hbar}\frac{1}{4}V_0\left(e^{+2ikx}+e^{-2ikx}+2\right)\\
  &=
  \frac{1}{4}V_0\left(2\delta_{m,n}+\delta_{m,n+2}+\delta{m,n-2}\right)
\end{align*}
this yields us finally
\begin{align*}
  \hat{H}_{mn}
  =
  \mel**{p_m}{\hat{H}}{p_n}
  &=
  \left((n+q/k)^2\frac{\hbar^2k^2}{2m}+\frac{1}{2}V_0\right)\delta_{n,m}
  +\frac{1}{4}V_0\left(\delta_{m,n+2}+\delta_{m,n-2}\right)\\
  &=
  \begin{cases}
    (n+q/k)^2\frac{\hbar^2k^2}{2m}+\frac{1}{2}V_0, & \text{if } n=m\\
    \frac{1}{4}V_0, & \text{if } \abs{n-m}=2\\
    0, & \text{otherwise}
  \end{cases}
\end{align*}
where we can identify the recoil energy $E_r=(\hbar k)^2/(2m)$. The recoil
energy can be used to obtain an energy scale independent of the particular
atom sort used.
