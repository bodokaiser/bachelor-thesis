\chapter{Introduction}

Complex quantum systems studied in condensed matter physics, quantum chemistry
as well as high-energy physics are not always accessable in a manner as
required by research. Experiments involving ultracold atoms in optical
lattices allow the study of many body quantum systems in a controllable
environment, giving us the opportunity to simulate and explore quantum
effects and expand our current understanding of quantum mechanics and
statistical physics \cite{Gross2017}.

The central idea behind these types of experiments is to cool down neutral
atoms to (below) micro Kelvin and load them into an optical lattice. At these
low temperatures the atoms demonstrate quantum behaviour. The optical lattices
act as periodic potentials analogue to the periodic potential inside crystal
lattices as known from solid state physics \cite{Lewenstein2007}.

However, unlike the situation in solid states where we only have limited
prospects to alter a systems properties, lasers driven by state of the art
optics and electronics give us a wide range of well controllable parameters.

One of these parameters of interest is the ability to create local potentials
and coincidentally topic of the present work. Local potentials can be used to
but are not limited to simulate impurities in solid states.

\section{Related Work}

% TODO: Reference paris paper

Local manipulations of atoms inside optical lattices have been known for some
time in the embodiment of optical tweezers which allow trapping, stacking and
sorting of particles \cite{Tadmor2014}. Yet only recently attempts have been
reported to interact with local particle clusters through high-precision
time-averaged optical potentials \cite{Roy2016}.

In the present work we want to expand on said concept of local potentials in
that we demonstrate an optical setup which allows for these.
