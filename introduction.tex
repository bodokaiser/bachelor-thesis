\chapter{Introduction}

Complex quantum systems studied in condensed matter physics, quantum chemistry
as well as high-energy physics are not always accessable in a manner
required by research. Experiments involving ultracold atoms in optical
lattices allow the study of many body quantum systems in a controllable
environment, giving us the opportunity to simulate and explore quantum
effects and expand our current understanding of quantum mechanics and
statistical physics \cite{Gross2017}.

The central idea behind these types of experiments is to cool down neutral
atoms to (below) micro Kelvin and load them into an optical lattice. At these
low temperatures the atoms demonstrate quantum behaviour. The optical lattices
act as periodic potentials analogue to the periodic potential found inside
solid state crystal lattices \cite{Lewenstein2007}.

However, unlike the situation in solid states where we only have limited
prospects to alter a systems properties, lasers driven by state of the art
optics and electronics give us a wide range of well controllable parameters.

One of the parameters of interest is the ability to apply local potentials
to the system which for example can be used to simulate impurities in solid
bodies.

\section{Related Work}

% TODO: Reference paris paper

Local manipulations of atoms inside optical lattices have been known for some
time in the embodiment of optical tweezers that allow trapping, stacking and
sorting of particles \cite{Tadmor2014}. Yet, only recently attempts to
interact with local particle clusters through high-precision time-averaged
optical potentials have been reported \cite{Roy2016}.

In the following we continue on the laid out work \cite{Hertlein2017} which
provided us with an optical setup for single-site manipulation using
acousto-optic deflectors as well as considerations with regard to
aperture limited gaussian beam propagation.

